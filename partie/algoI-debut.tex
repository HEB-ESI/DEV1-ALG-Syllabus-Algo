% =======================================================
% Syllabus de Logique 1ère - Début du document
% =======================================================

% =======================================================
% Page de garde
% =======================================================

\thispagestyle{empty}

% haut-gauche
\includegraphics[scale=0.45]{image/logo-esi}
\begin{minipage}[t]{7cm}
\vspace{-6.5\baselineskip}
\sffamily
\large\textbf{\ecole\\\entite}
\\\vspace{3.5mm}\\
\large\entiteadresse\\\entitetel{} – \entitemail
\end{minipage}
%
% haut-droit
\begin{minipage}[t]{5cm}
\vspace{-6.5\baselineskip}
\sffamily
\raggedleft
\large\textbf{\etude}
\end{minipage}


% centre
\vfill
\begin{center}
\sffamily
\Huge\cours
\bigskip\\
\Large\ue\ -- \annee
\end{center}
\vfill

%bas
Activité d'apprentissage enseignée par :
\begin{center}
\itshape 
\begin{tabular}{*{5}{p{2.2cm}}}
À définir\dots
%\auteura & \auteurb & \auteurc & \auteurd & \auteure \\
%\auteurf & \auteurg & \auteurh & \auteuri & \auteurj \\
\end{tabular}
\end{center}

% =======================================================
% 2ème page : licence, infos de version, ...
% =======================================================

\clearpage
\thispagestyle{empty}

\vfill

Ce syllabus a été écrit à l'origine par M. Monbaliu
à une époque où le cours s'appelait 
\og\ Logique et techniques de programmation \fg.
Il a ensuite été adapté par Mme~Leruste, M.~Beeckmans et M.~Codutti.
Qu'ils en soient tous remerciés.
Nous remercions également tous ceux qui ont contribué à son amélioration
grâce à leur lecture attentive et leurs remarques. 

\bigskip
Document produit avec \LaTeX.
\\Version du \today.

\vfill

\includegraphics[width=25mm]{image/cc-gris}
\\
Ce document est distribué sous licence 
\\Creative Commons Paternité - Partage à l'Identique 2.0 Belgique 
\\(http://creativecommons.org/licenses/by-sa/2.0/be/).
\\Les autorisations au-delà du champ de cette licence
\\peuvent être obtenues à \entitesite{} - \texttt{\contact}.
\pagestyle{fancy}

% =======================================================
% Table des matières
% =======================================================
\setcounter{tocdepth}{1}
\tableofcontents{}

