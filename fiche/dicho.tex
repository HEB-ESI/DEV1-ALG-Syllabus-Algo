%================================
\begin{Fiche}{Recherche dichotomique}
%================================
\label{fiche:dicho}

	Trouver rapidement la position d’une valeur donnée dans un tableau
	\textbf{trié} d’entiers. 
	Si la valeur n’est pas présente, 
	on donnera la position où elle aurait du se trouver.

\Section{Spécification}
	
	\textbf{Données}~: 
		\begin{itemize}
		\item le tableau à analyser
		\item la valeur recherchée
		\end{itemize}
		
	\textbf{Résultat}~:
		\begin{itemize}
		\item un booléen indiquant si la valeur a été trouvée
		\item un entier indiquant
			soit la position où la valeur a été trouvée
			soit la position où elle aurait du être.
		\end{itemize}

\Section{Solution}

		L’algorithme rapide que nous avons vu est la recherche
		dichotomique.
		
		\begin{LDA}
			\Algo{rechercheDichotomique}{
					\\\hfill
					\Par{tab\In}{\Array{n}{entiers}}, 
					\Par{valeur\In}{entier}, 
					\Par{pos\Out}{entier}
					}{booléen}
				\Decl{indiceDroit, indiceGauche, indiceMédian}{entiers}
				\Decl{candidat}{entier}
				\Decl{trouvé}{booléen}
				\Empty
				\Let indiceGauche \Gets 0
				\Let indiceDroit \Gets n-1
				\Let trouvé \Gets faux
				\Empty
				\While{NON trouvé ET indiceGauche {${\leq}$} indiceDroit}
					\Let indiceMédian \Gets (indiceGauche + indiceDroit) DIV 2
					\Let candidat \Gets tab[indiceMédian]
					\If{candidat = valeur} 
						\Let trouvé \Gets vrai
					\ElsIf{candidat < valeur}
						\Let indiceGauche \Gets indiceMédian + 1
						\RComment on garde la partie droite
					\Else
						\Let indiceDroit \Gets indiceMédian – 1
						\RComment on garde la partie gauche
					\EndIf
				\EndWhile
				\Empty
				\If{trouvé}
					\Let pos \Gets indiceMédian
				\Else
					\Let pos \Gets indiceGauche
					\RComment dans le cas où la valeur n’est pas trouvée,
					\Empty 
					\RComment on vérifiera que indiceGauche donne la valeur où elle pourrait être insérée.
				\EndIf
				\Empty
				\Return trouvé
			\EndAlgo
		\end{LDA}	
	
\end{Fiche}
