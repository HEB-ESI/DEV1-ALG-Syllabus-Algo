%================================
\begin{Fiche}{Parcours partiel d'un tableau}
%================================
\label{fiche:tab-parcours-partiel}

	Recherche d'un zéro dans un tableau.

\Section{Spécification}
	
	\textbf{Données} : le tableau à tester
		
	\textbf{Résultat} : 
		un booléen à vrai si il existe une valeur nulle dans le tableau.
	
	\begin{center}	
		\flowalgod{tab (tableau d'entiers)}{contientZéro}{booléen}
	\end{center}

\Section{Solution}

	Contrairement au parcours complet 
	(cf. fiche \vref{fiche:tab-parcours-complet})
	on va utiliser un \emph{tant que}
	car on peut s'arrêter dès qu'on trouve ce qu'on cherche.
	
	Il existe essentiellement deux solutions, avec ou sans variable booléenne.
	En général, la solution [A] sera plus claire si le test est court.

	\subsubsection*{[A] Sans variable booléenne}
	
		\begin{LDA}
			\Algo{contientZéro}{\Par{tab}{\Array{n}{entiers}}}{booléen}
				\Decl{i}{entier}
				\Let i \Gets 0
				\While{i < n ET tab[i] $\neq$ 0}
					\Let i \Gets i + 1
				\EndWhile
				\Return i < n \RComment Si i<n -> arrêt prématuré -> on a trouvé un 0.
			\EndAlgo
		\end{LDA}
	
		Il faut être attentif à \textbf{ne pas inverser}
		les deux parties du test.
		Il faut absolument vérifier que l'indice est bon avant
		de tester la valeur à cet indice.
		Revoyez la notion de court-circuit (\vref{court-circuit} ).
		
	\subsubsection*{[B] Avec variable booléenne}

		\begin{LDA}
			\Algo{contientZéro}{\Par{tab}{\Array{n}{entiers}}}{booléen}
				\Decl{i}{entier}
				\Decl{zéroPrésent}{booléen}
				\Let zéroPrésent \Gets faux
				\Let i \Gets 0
				\While{i < n ET NON zéroPrésent}
					\Let zéroPrésent \Gets tab[i] = 0
					\Let i \Gets i + 1
				\EndWhile
				\Return zéroPrésent
			\EndAlgo
		\end{LDA}

		Au sortir de la boucle,
		l'indice \lda{i} ne désigne pas l'élément
		qui nous a permis d'arrêter mais le suivant.
		Si nécessaire, on peut remplacer l'intérieur de la boucle par :

		\begin{LDA}
			\If{tab[i] = 0}
				\Let zéroPrésent \Gets vrai
			\Else
				\Let i \Gets i + 1
			\EndIf
		\end{LDA}
		
		Dans notre exemple, 
		on cherche un élément particulier (un 0).
		Dans le cas où on vérifie si tous les éléments possèdent 
		une certaine propriété (être positifs par exemple),
		on veillera à adapter le nom du booléen et son utilisation
		(par exemple un booléen appelé \lda{tousPositifs},
		initialisé à vrai avec un \lda{\dots ET tousPositifs}
		dans le test.

		Dans tous les cas,
		faites attention à ne pas utiliser \lda{tab[i]}
		si vous n'êtes pas sûr du \lda{i}.
		C'est particulièrement vrai après la boucle.
		
\Section{Alternatives}

	\begin{wrapfigure}{l}{90mm}
		\begin{LDA}
			\Algo{tousPositifs}{
				\Par{tab}{\Array{n}{entiers}}}{booléen}
				\For{i}{0}{n-1}
					\If{tab[i] = 0}
						\Return faux
					\EndIf
				\EndFor
				\Return vrai
			\EndAlgo
		\end{LDA}
	\end{wrapfigure}

	On rencontre parfois ce genre de solutions.
	Elle est plus simple à écrire car on ne doit
	pas gérer manuellement l'indice
	mais le lecteur doit attendre de lire
	le code de la boucle pour comprendre
	que le parcours ne sera pas toujours complet.
	
	Nous ne la recommandons pas en 1ère année
	mais nous l'accepterons 
	s'il s'agit d'un algorithme court
	qui ne contient que ce parcours.
					
\Section{Quand l'utiliser ?}

	Ce type de solution peut être utilisé à chaque fois
	qu'on parcourt un tableau mais qu'un arrêt avant la fin est possible.

	\begin{itemize}
	\item Est-ce que tous les éléments sont positifs ?
	\item Est-ce que les élément sont triés ?
	\item Est-ce qu'un élément précis est présent ?
	\item \dots
	\end{itemize}
	
\end{Fiche}
