\begin{Fiche}{Un calcul simple}
\label{fiche:calcul-simple}

\Section{Le problème}
	Calculer la surface d'un rectangle 
	à partir de sa longueur et sa largeur.

\Section{Spécification}

	\paragraph{Données}
	\begin{itemize}
	\item La longueur du rectangle ;
	\item sa largeur.
	\end{itemize}
	Toutes les données sont des réels positifs ou nuls.

	\paragraph{Résultat.}
	Un réel représentant la surface du rectangle

	\bigskip
	\begin{center}	
	\flowalgodd{longueur (réel)}{largeur (réel)}{surfaceRectangle}{réel}
	\end{center}

\Section{Exemples}

	\begin{itemize}
	\item \lda{surfaceRectangle(4, 3)} donne $12$
	\item \lda{surfaceRectangle(2.5, 2)} donne $5$
	\end{itemize}

\Section{Analyse de la solution}

	La surface d'un rectangle est obtenue en multipliant
	la largeur par la longueur.
	\begin{equation}
		\textrm{surface} = \textrm{longueur} * \textrm{largeur}
	\end{equation}
	
\Section{Solution}

	\begin{LDA}
	%\LComment Calcule la surface du rectangle
	%\LComment Données : la longueur et la largeur du rectangle
	%\LComment Résultat : la surface du rectangle
	%mcd: je trouve çà un peu bateau à ce stade
	\Algo{surfaceRectangle}{\Par{longueur, largeur}{réel}}{réel}
		\Return longueur * largeur
	\EndAlgo
	\end{LDA}

\Section{Quand l'utiliser ?}

	Ce type de solution peut être utilisé à chaque fois
	que la réponse s'obtient par un calcul simple sur les données.
	Si le calcul est plus complexe, 
	il peut être utile de le décomposer pour accroitre la lisibilité
	(cf. fiche \vref{fiche:calcul-complexe}) 
	
\end{Fiche}
