%================================
\begin{Fiche}{Parcours complet d’un tableau}
%================================
\label{fiche:tab-parcours-complet}

	Afficher tous les éléments d’un tableau d’entiers.

\Section{Spécification}
	
	\begin{itemize}
	\item \textbf{Données}~: le tableau à afficher
	\item \textbf{Résultat}~: aucun.
	\item \textbf{Affiche} : les éléments du tableau, dans l'ordre.
	\end{itemize}

\Section{Solution}

	Puisqu’on parcourt tout le tableau,
	on peut utiliser une boucle \emph{pour}.
	
	\begin{LDA}
		\Algo{afficherTab}{\Par{tab}{\Array{n}{entiers}}}{}
			\For{i}{0}{n-1}
				\Write tab[i]
			\EndFor
		\EndAlgo
	\end{LDA}

\Section{Quand l’utiliser~?}

	Ce type de solution peut être utilisé à chaque fois
	qu’on doit examiner \textbf{tous} les éléments d’un tableau,
	quel que soit le traitement qu’on en fait~:
	les afficher, les sommer, les comparer\dots
	
	
\end{Fiche}
