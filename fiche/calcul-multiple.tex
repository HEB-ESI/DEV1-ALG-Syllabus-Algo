\begin{Fiche}{Un nombre (im)pair}
\label{fiche:calcul-multiple}

\Section{Le problème}
	Un nombre reçu en paramètre est-il un multiple d’un autre~?

\todo {modifier la suite qui est un copier-coller de pair}
\Section{Analyse}

	Un nombre est pair si il est multiple de 2. 
	C’est-à-dire si 	le reste de sa division par 2 vaut 0.
	\begin{LDA}
		\Stmt estPair est vrai si nombre MOD 2 = 0
	\end{LDA}

	\paragraph{Données}
	\begin{itemize}
	\item le nombre dont on veut savoir si il est pair.
	\end{itemize}
	Un entier.

	\paragraph{Résultat.}
	Un booléen à \textit{vrai} si le \textit{nombre} est pair et \textit{faux} sinon.

	\bigskip
	\begin{center}	
	\SchemaModulei{nombre}{isPair}{booléen}
	\end{center}

\Section{Exemples}

	\begin{itemize}
	\item \lda{isPair(2016)} donne $vrai$
	\item \lda{isPair(2015)} donne $faux$
	\end{itemize}
	
\Section{Solution}

	\begin{LDA}
	\LComment Retourne vrai si le nombre reçu en paramètre est pair et faux sinon.
	\LComment Données~: le nombre dont on veut savoir si il est pair
	\LComment Résultat~: vrai si le nombre est pair et faux sinon.
	\Module{isPair}{\Par{nombre}{entier}}{booléen}
		\Return nombre MOD 2 = 0
	\EndModule
	\end{LDA}

\Section{Quand l’utiliser~?}

	\todo{todo}
	
\Section{Fiches liées}
	
	\begin{itemize}
	\item
		La fiche \vref{fiche:calcul-pair} présente
		une solution plus générale.
	\end{itemize}
	
\end{Fiche}
