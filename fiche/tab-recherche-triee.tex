%================================
\begin{Fiche}{Tableau trié}
%================================
\label{fiche:tab-recherche-triee}

	Rechercher, ajouter, supprimer des données triées dans un tableau d’entiers.

%\Section{Spécification}
	
	%\textbf{Données}~: le tableau à analyser
		
	%\textbf{Résultat}~: la valeur du maximum

\Section{Solution}

	Rechercher la position où a été trouvé l’élément ou la position où il aurait dû être
	
	\textbf{Données}~: le tableau à analyser, le nombre d'éléments dans ce tableau, la valeur à rechercher
		
	\textbf{Résultat}~: la position de l'élément si il est dans le tableau et -1 sinon

		\begin{LDA}
				\LComment{Recherche un étudiant.}
				\LComment{- trouvé~: indique si oui ou non il a été trouvé}
				\LComment{- pos~: indique la position où a été trouvé l’étudiant ou la position où il aurait dû être}
				\Algo{rechercher}{
						\Par{tab\In}{\Array{n}{entiers}}, 
						\Par{nbElem\In}{entier}, 
						\\\hfill\Par{nbRecherché\In}{entier},
						\Par{trouvé\Out}{booléen},
						\Par{pos\Out}{entier}
						}{}
					\Let pos \Gets 0
					\While{pos < nbElem ET tab[pos] < nbRecherché}
						\Let pos \Gets pos + 1
					\EndWhile
					\Let trouvé \Gets pos < nbElem ET tab[pos] = nbRecherché
				\EndAlgo
			\end{LDA}

	Rechercher l'indice d'une donnée trouvée dans un tableau trié ou -1 si elle n'est pas trouvée
	
	\textbf{Données}~: le tableau à analyser, le nombre d'éléments dans ce tableau, la valeur à rechercher
		
	\textbf{Résultat}~: la position de l'élément si il est dans le tableau et -1 sinon
	
	Cette opération est triviale.
			
		\begin{LDA}
			\LComment{Vérifie si un nombre est dans un tableau d'entiers trié et donne sa position (-1 si non inscrit)}
			\Algo{vérifier}{
				\Par{tab\In}{\Array{n}{entiers}}, 
				\Par{nbElem\In}{entier}, 
				\\\hfill
				\Par{nbRecherché\In}{entier}
			}{entier}
				\Decl{pos}{entier}
				\Decl{trouvé}{booléen}
				\Stmt rechercher( tab, nbElem, nbRecherché, trouvé, pos )
				\If{trouvé}
					\Return pos
				\Else
					\Return -1
				\EndIf
			\EndAlgo
		\end{LDA}

	Ajouter une donnée non encore présente dans le tableau de données non triées
	
	\textbf{Données}~: le tableau à modifier, le nombre d'éléments dans ce tableau, la valeur à ajouter
		
	\textbf{Résultat}~: le tableau reçu est modifié en lui ajoutant la valeur si elle n'y était pas déjà
	
		\begin{LDA}
				\LComment{Ajouter un nombre donné.}
				\Algo{ajouter}{
						\Par{tab\In\Out}{\Array{n}{entiers}}, 
						\Par{nbElem\In\Out}{entier}, 
						\Par{nbAjouter\In}{entier}
						}{}
					\Decl{pos}{entier}
					\Decl{trouvé}{booléen}
					\Stmt rechercher( tab, nbElem, nbAjouter, trouvé, pos )
					\Stmt décalerDroite( tab, pos, nbElem)
					\Let tab[pos] \Gets nbAjouter
					\Let nbElem \Gets nbElem + 1
				\EndAlgo
				\Empty
				\LComment{Décale d’une position à droite les éléments entre la position début et fin}
				\Algo{décalerDroite}{
						\Par{tab\In\Out}{\Array{n}{entiers}}, 
						\Par{début\In}{entier}, 
						\Par{fin\In}{entier}
						}{}
					\For[-1]{i}{fin}{début}
						\Let tab[i+1] \Gets tab[i]
					\EndFor
				\EndAlgo
			\end{LDA}
	
	Supprimer une donnée d'un tableau de données non triées
	
	\textbf{Données}~: le tableau à modifier, le nombre d'éléments dans ce tableau, la valeur à supprimer
		
	\textbf{Résultat}~: le tableau reçu est modifié en lui supprimant la valeur
		
		\begin{LDA}
			\LComment{supprimer l'élément donné}
			\Algo{supprimer}{
				\Par{tab\In\Out}{\Array{n}{entiers}}, 
				\Par{nbElem\In\Out}{entier}, 
				\Par{nbSupprimer\In}{entier}
			}{}
				\Decl{pos}{entier}
				\Decl{trouvé}{booléen}
				\Stmt rechercher( tab, nbElem, nbSupprimer, trouvé, pos )
				\Stmt décalerGauche( tab, pos+1, nbElem)
				\Let nbElem \Gets nbElem - 1					
			\EndAlgo
			\Empty
			\LComment{Décale d’une position à gauche les éléments entre la position début et fin}
			\Algo{décalerGauche}{
				\Par{tab\In\Out}{\Array{n}{entiers}}, 
				\Par{début\In}{entier}, 
				\Par{fin\In}{entier}
			}{}
				\For{i}{début}{fin}
					\Let tab[i-1] \Gets tab[i]
				\EndFor
			\EndAlgo
			\end{LDA}

%\Section{Variante}

\end{Fiche}