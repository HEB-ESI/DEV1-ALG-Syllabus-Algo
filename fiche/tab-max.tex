%================================
\begin{Fiche}{Maximum dans un tableau}
%================================
\label{fiche:tab-max}

	Trouver la valeur maximale dans un tableau d'entiers.

\Section{Spécification}
	
	\textbf{Données} : le tableau à analyser
		
	\textbf{Résultat} : la valeur du maximum

\Section{Solution}

	Il faut veiller à initialiser le maximum
	avec la première valeur du tableau
	pour parcourir le reste du tableau à la recherche d'une valeur plus grande.
	
	\begin{LDA}
		\Algo{max}{\Par{tab}{\Array{n}{entiers}}}{entier}
			\Decl{max}{entier}
			\Let max \Gets tab[0]
			\For{i}{1}{n-1}
				\If{tab[i]>max}
					\Let max \Gets tab[i]
				\EndIf
			\EndFor
			\Return max
		\EndAlgo
	\end{LDA}

\Section{Variante}

	On peut être intéressé par la position où se trouve le
	(en tout cas un) maximum.

	\begin{LDA}
		\Algo{max}{\Par{tab}{\Array{n}{entiers}}}{entier}
			\Decl{posMax}{entier}
			\Let posMax \Gets 0
			\For{i}{1}{n-1}
				\If{tab[i]>tab[posMax]}
					\Let posMax \Gets i
				\EndIf
			\EndFor
			\Return posMax
		\EndAlgo
	\end{LDA}
	
	
\end{Fiche}
