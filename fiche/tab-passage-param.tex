%================================
\begin{Fiche}{Passage d’un tableau en paramètre}
%================================
\label{fiche:tab-passage-param}

	Le type \emph{tableau} étant un type à part entière,
		il est tout-à-fait éligible comme type
		pour les paramètres et la valeur de retour d’un algorithme.

%\Section{Spécification}
	

%\Section{Solution}
	\section{Passer un tableau en paramètre}
		\begin{itemize}
			\item \In : 
				indique que l’algorithme va consulter les valeurs 
				du tableau reçu en paramètre. 
				Les éléments doivent donc avoir été initialisés
				avant d’appeler l’algorithme. Exemple~:
			
				\begin{LDA}
					\LComment{Affiche les éléments d’un tableau de n entiers}
					\Algo{afficher}{\Par{tab\In}{\Array{n}{entiers}}}{} 
						\For{i}{0}{n-1}
							\Write tab[i]
						\EndFor
					\EndAlgo 
		
					\Empty
					\LComment{Utilisation possible}
					\Decl{monTab}{\Array{10}{entiers}}
					\Let monTab \Gets \{2,3,5,7,11,13,17,19,23,29\}
					\Stmt afficher(monTab)
				\end{LDA}
				
				Rappelons qu’il s’agit du passage
				par défaut si aucune flèche n’est indiquée.
				
			\item \In\Out :
				indique que l’algorithme va consulter/modifier les valeurs 
				du tableau reçu en paramètre. Exemple~:
			
				\begin{LDA}
					\LComment{Inverse le signe des éléments d'un tableau de n entiers}
					\Algo{inverserSigne}{\Par{tab\In\Out}{\Array{n}{entiers}}}{} 
						\For{i}{0}{n-1}
							\Let tab[i] \Gets -tab[i]
						\EndFor
					\EndAlgo 
		
					\Empty
					\LComment{Utilisation possible}
					\Decl{monTab}{\Array{5}{entiers}}
					\Let monTab \Gets \{2,-3,5,-7,11\}
					\Stmt inverserSigne(monTab)
				\end{LDA}
	
			\item \Out :
				indique que l’algorithme va assigner des valeurs 
				au tableau reçu en paramètre.
				Les éléments de ce tableau n’ont donc pas à être initialisés
				avant d’appeler cet algorithme. Exemple~:
			
				\begin{LDA}
					\LComment{Initialise un tableau de n entiers reçu en paramètre}
					\Algo{initialiser}{\Par{tab\Out}{\Array{n}{entiers}}}{}
						\For{i}{0}{n-1}
							\Let tab[i] \Gets i
						\EndFor
					\EndAlgo 
					\Empty
					\LComment{Utilisation possible}
					\Decl{monTab}{\Array{n}{entiers}}
					\Stmt initialiser(monTab)
				\end{LDA}

			\end{itemize}

		\section{Retourner un tableau}
		%---------------------------------
			
			Comme pour n’importe quel autre type,
			un algorithme peut retourner un tableau.
			Ce sera à lui de le déclarer et de lui donner des valeurs.

			\textbf{Exemple~:}
			\begin{LDA}
				\LComment Crée un tableau d’entiers de taille n, l’initialise à 0 et le retourne.
				\Algo{créer}{n : entier}{\Array{n}{entiers}}
					\Decl{tab}{\Array{n}{entiers}}
					\For{i}{0}{n-1}
						\Let tab[i] \Gets 0
					\EndFor
					\Return tab
				\EndAlgo
				\Empty
				\LComment{Utilisation possible}
				\Algo{test}{}{}
					\Decl{entiers}{\Array{20}{entiers}}
					\Let entiers \Gets créerTableau(20)
					\Stmt afficher(entiers)
				\EndAlgo
			\end{LDA}
%\Section{Variante}
	
\end{Fiche}
