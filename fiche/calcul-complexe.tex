%================================
\begin{Fiche}{Un calcul complexe}
%================================
\label{fiche:calcul-complexe}

	Calculer la vitesse (en km/h) d'un véhicule dont on donne
	la durée du parcours (en secondes) 
	et la distance parcourue (en mètres).

\Section{Spécification}
	
	\textbf{Données} (toutes réelles et non négatives) :
		\begin{itemize}
		\item la distance parcourue par le véhicule (en m) ;
		\item la durée du parcours (en s).
		\end{itemize}
		
	\textbf{Résultat} : la vitesse du véhicule (en km/h).

	\begin{center}
	\flowalgodd{distanceM (réel)}{duréeS (réel)}{vitesseKMH}{réel}
	\end{center}

\Section{Exemples}

	\begin{itemize}
	\item \lda{vitesseKMH(100,10)} donne $36$
	\item \lda{vitesseKMH(10000,3600)} donne $10$
	\end{itemize}

\Section{Solution}

	La vitesse est liée à la distance et à la durée par la formule :
	\[
		\textrm{vitesse} = \frac{\textrm{distance}}{\textrm{durée}}
	\]

	pour autant que les unités soient cohérentes.
	Ainsi pour obtenir une vitesse en km/h, 
	il faut convertir la distance en kilomètres 
	et la durée en heures.
	Ce qui donne :
		
	\begin{LDA}
		\Algo{vitesseKMH}{\Par{distanceM, duréeS}{réel}}{réel}
			\Decl{distanceKM, duréeH}{réel}
			\Let distanceKM \Gets $\frac{\textrm{distanceM}}{1000}$
			\Let duréeH \Gets $\frac{\textrm{duréeS}}{3600}$
			\Return $\frac{\textrm{distanceKM}}{\textrm{duréeH}}$
		\EndAlgo
	\end{LDA}

\Section{Vérification}

	\begin{center}
		\begin{tabular}{|c|cccc|c|}
		\hline
		test \no & distance en mètres & durée en secondes & réponse attendue & réponse fournie & {} \\\hline
		\hline 
		1 & 100   & 10   & 36 & 36 & {\color{ForestGreen}$\checkmark$} \\\hline
		2 & 10000 & 3600 & 10 & 10 & {\color{ForestGreen}$\checkmark$} \\\hline
		\end{tabular}
	\end{center}								

\Section{Quand l'utiliser ?}

	Ce type de solution peut être utilisé à chaque fois
	que la réponse s'obtient par un calcul complexe sur les données
	qu'il est bon de décomposer pour aider à sa lecture.
	Si le calcul est plutôt simple, 
	on peut le garder en une seule assignation
	(cf. fiche \vref{fiche:calcul-simple}).
	
\end{Fiche}
