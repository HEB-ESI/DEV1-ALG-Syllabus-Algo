\begin{Fiche}{Un calcul complexe}
\label{fiche:calcul-complexe}

\Section{Le problème}
	Calculer la vitesse (en km/h) d'un véhicule dont on donne
	la durée du parcours (en secondes) 
	et la distance parcourue (en mètres).

\Section{Spécification}

	\paragraph{Données}
	\begin{itemize}
	\item la distance parcourue par le véhicule (en m);
	\item la durée du parcours (en s).
	\end{itemize}
	Toutes les données sont des réels

	\paragraph{Résultat.}
	Un réel représentant la vitesse du véhicule (en km/h).

	\bigskip
	\begin{center}	
	\flowalgodd{distanceM (réel)}{duréeS (réel)}{vitesseKMH}{réel}
	\end{center}

\Section{Exemples}

	\begin{itemize}
	\item \lda{vitesseKMH(100,10)} donne $36$
	\item \lda{vitesseKMH(10000,3600)} donne $10$
	\end{itemize}

\Section{Analyse de la solution}

	La vitesse est liée à la distance et à la durée par la formule :
	\begin{equation}
		\textrm{vitesse} = \frac{\textrm{distance}}{\textrm{durée}}
	\end{equation}

	pour autant que les unités soient cohérentes.
	Ainsi pour obtenir une vitesse en km/h, 
	il faut convertir la distance en kilomètres 
	et la durée en heures.
		
\Section{Solution}

	\begin{LDA}
%	\LComment Calcule la vitesse d'un véhicule en connaissant la distance et la durée
%	\LComment Données : la distance parcourue en m et le temps écoulé en secondes
%	\LComment Résultat : la vitesse du véhicule en km/h
	\Algo{vitesseKMH}{\Par{distanceM, duréeS}{réel}}{réel}
		\Decl{distanceKM, duréeH}{réel}
		\Let distanceKM \Gets $\frac{\textrm{distanceM}}{1000}$
		\Let duréeH \Gets $\frac{\textrm{duréeS}}{3600}$
		\Return $\frac{\textrm{distanceKM}}{\textrm{duréeH}}$
	\EndAlgo
	\end{LDA}

\Section{Quand l'utiliser ?}

	Ce type de solution peut être utilisé à chaque fois
	que la réponse s'obtient par un calcul complexe sur les données
	qu'il est bon de décomposer pour aider à sa lecture.
	Si le calcul est plutôt simple, 
	on peut le garder en une seule assignation
	(cf. fiche \vref{fiche:calcul-simple}).
	
\end{Fiche}
