%===================================
\begin{Fiche}{Un nombre pair}
%===================================
\label{fiche:calcul-pair}

	Un nombre reçu en paramètre est-il pair~?

\Section{Spécification}

	\textbf{Données} : le nombre entier dont on veut savoir si il est pair.
		
	\textbf{Résultat} : un booléen à \textit{vrai} si le \textit{nombre} est pair et \textit{faux} sinon.

	\begin{center}	
		\flowalgod{nombre (entier)}{estPair}{booléen}
	\end{center}

\Section{Exemples}

	\begin{itemize}
	\item \lda{estPair(2016)} donne $vrai$
	\item \lda{estPair(2015)} donne $faux$
	\end{itemize}
	
\Section{Solution}

	Un nombre est pair si il est multiple de 2. 
	C'est-à-dire si le reste de sa division par 2 vaut 0.
	%\[
		%estPair\ est\ vrai\ \equiv nombre\ MOD\ 2 = 0
	%\]

	\begin{LDA}
		\Algo{estPair}{\Par{nombre}{entier}}{booléen}
			\Return nombre MOD 2 = 0
		\EndAlgo
	\end{LDA}

	Attention à éviter les mauvaises écritures 
	expliquées à l'annexe \vref{B-ass-bool} :
	pas de \lda{\K{si-sinon}}.
	
%\Section{Alternatives}

	%\begin{minipage}[t]{7.5cm}
		%\begin{LDA}
		%\Algo{estPair}{\Par{nombre}{entier}}{booléen}
			%\If{nombre MOD 2 = 0}
				%\Return vrai
			%\Else
				%\Return faux
			%\EndIf
		%\EndAlgo
		%\end{LDA}
	%\end{minipage}
	%\quad
	%\begin{minipage}[t]{6cm}
		%Certains étudiants se sentent plus à l'aise avec
		%la solution ci-contre en début d'année.			
		%C'est probablement parce qu'elle colle plus à la façon de l'exprimer
		%en français.
		%On les encourage toutefois à rapidement 
		%passer à la version plus compacte
		%et, une fois habitué, plus lisible.
		%Cette alternative sera sancitonnée dans les évaluations.
	%\end{minipage}

	%\hfil\rule{0.5\textwidth}{.4pt}\hfil

	%\begin{minipage}[t]{7.5cm}
		%\begin{LDA}
		%\Algo{estPair}{\Par{nombre}{entier}}{booléen}
			%\Decl{pair}{booléen}
			%\If{nombre MOD 2 = 0}
				%\Let pair \Gets vrai
			%\Else
				%\Let pair \Gets faux
			%\EndIf
			%\Return pair
		%\EndAlgo
		%\end{LDA}
	%\end{minipage}
	%\quad
	%\begin{minipage}[t]{6cm}
		%Cette solution n'est pas non plus à privilégier.
		%On a un seul retour, ce qui est mieux,
		%mais le test est encore de trop.
		%Nous la sanctionnerons dans les évaluations.
	%\end{minipage}

	%\hfil\rule{0.5\textwidth}{.4pt}\hfil

	%\begin{minipage}[t]{7.5cm}
		%\begin{LDA}
		%\Algo{estPair}{\Par{nombre}{entier}}{booléen}
			%\If{nombre MOD 2 = 0}
				%\Return vrai
			%\EndIf
			%\Return faux
		%\EndAlgo
		%\end{LDA}
	%\end{minipage}
	%\quad
	%\begin{minipage}[t]{6cm}
		%On rencontre également ce genre de solution
		%qui, pour certains, 
		%parait mieux que la précédente 
		%parce qu'elle ne contient pas de "sinon"
		%et est donc plus courte.
		%Il n'en n'est rien.
		%Rappelons que la longueur de l'algorithme
		%n'est pas, en soi, un critère de qualité. 
		%À proscrire, donc.
	%\end{minipage}
	
\Section{Quand l'utiliser ?}

	À chaque fois qu'un résultat booléen dépend d'un calcul simple.
	Si le calcul est plus compliqué, on peut le décomposer comme
	indiqué dans la fiche \vref{fiche:calcul-complexe}.
	
	On peut également s'inspirer de cette solution
	quand il faut donner sa valeur à une variable booléenne.
		
\end{Fiche}
