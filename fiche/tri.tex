%================================
\begin{Fiche}{Tri d'un tableau}
%================================
\label{fiche:tri}

	Trier un tableau d'entiers par ordre croissant.

\Section{Spécification}
	
	\textbf{Données} : le tableau non trié, à trier.
		
	\textbf{Résultat} : ce même tableau trié.

\Section{Solution}

		Nous avons vu trois algorithmes

		\subsection*{Le tri par insertion}
		
	\begin{LDA}
		\LComment Trie le tableau reçu en paramètre (via un tri par insertion).
		\Algo{triInsertion}{\Par{tab\InOut}{\Array{n}{entiers}}}{}
			\Decl{i, j, valÀInsérer}{entiers}
			\For{i}{1}{n-1}
				\Let valÀInsérer \Gets tab[i]
				\LComment recherche de l’endroit où insérer valÀInsérer dans le 
				\LComment sous-tableau trié et décalage simultané des éléments
				\Let j \Gets i - 1
				\While{j ${\geq}$ 0 ET valÀInsérer < tab[j]}
					\Let tab[j+1] \Gets tab[j]
					\Let j \Gets j – 1
				\EndWhile
				\Let tab[j+1] \Gets valÀInsérer
			\EndFor
		\EndAlgo
	\end{LDA}

		\subsection*{Le tri par sélection des minima successifs}
		
	\begin{LDA}
		\LComment Trie le tableau reçu en paramètre (via un tri par sélection des minima successifs).
		\Algo{triSélectionMinimaSuccessifs}{\Par{tab\InOut}{\Array{n}{entiers}}}{}
			\Decl{i, indiceMin}{entier}
			\For{i}{0}{n – 2}
			\RComment i correspond à l’étape de l’algorithme
				\Let indiceMin \Gets positionMin( tab, i, n-1 )
				\Stmt swap( tab[i], tab[indiceMin] )
			\EndFor
		\EndAlgo
	\end{LDA}

	\begin{LDA}
		\LComment Retourne l’indice du minimum entre les indices début et fin du tableau reçu.
		\Algo{positionMin}{\Par{tab\InOut}{\Array{n}{entiers}}, Par{début, fin}{entiers}}{entier}
			\Decl{indiceMin, i}{entiers}
			\Let indiceMin \Gets début
			\For{i}{début+1}{fin}
				\If{tab[i] < tab[indiceMin]}
					\Decl indiceMin \Gets i
				\EndIf
			\EndFor
			\Return indiceMin
		\EndAlgo
	\end{LDA}

	\begin{LDA}
		\LComment {Échange le contenu de 2 variables.}
		\Algo{swap}{\Par{a\InOut, b\InOut}{entiers}}{}
			\Decl{aux}{entiers}
			\Let aux \Gets a
			\Let a \Gets b
			\Let b \Gets aux
		\EndAlgo
	\end{LDA}

		\subsection*{Le tri bulle}

	\begin{LDA}
	\LComment Trie le tableau reçu en paramètre (via un tri bulle).
		\Algo{triBulle}{\Par{tab\InOut}{\Array{n}{entiers}}}{}
			\Decl{indiceBulle, i}{entiers}
			\For{indiceBulle}{1}{n-1}
				\For[-1]{i}{n – 1}{indiceBulle}
					\If{tab[i] > tab[i + 1]}
						\Stmt swap( tab[i], tab[i + 1] )
						\RComment voir algorithme précédent
					\EndIf
				\EndFor
			\EndFor
		\EndAlgo
	\end{LDA}


	
\end{Fiche}
