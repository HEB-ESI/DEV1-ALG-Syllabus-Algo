\chapter{Aide-mémoire}

Cet aide-mémoire peut vous accompagner lors d'une
interrogation ou d'un examen. 
Il vous est permis d’utiliser ces méthodes sans les développer.
Par contre, si vous sentez le besoin d’utiliser 
une méthode qui n'apparait pas ici, 
il faudra en écrire explicitement le contenu.

\section{Pour manipuler les chaines et les caractères}

\begin{Pseudocode}
	\Stmt {\large // Est-ce ?}
	\Empty
	\Stmt estLettre(car: caractère) \Gives~booléen		\RComment{est-ce une lettre ?}
	\Stmt estChiffre(car: caractère) \Gives~booléen		\RComment{est-ce un chiffre ?}
	\Stmt estMajuscule(car: caractère) \Gives~booléen	\RComment{est-ce une majuscule ?}
	\Stmt estMinuscule(car: caractère) \Gives~booléen	\RComment{est-ce une minuscule ?}
	\Empty
	\Stmt {\large // Conversions}
	\Empty
	\Stmt majuscule(car: caractère) \Gives~caractère	\RComment{convertit une minuscule en une majuscule.}
	\Stmt minuscule(car: caractère) \Gives~caractère	\RComment{convertit une majuscule en une minuscule.}
	\Stmt numLettre(car: caractère) \Gives~entier		\RComment{donne la position de la lettre dans l'alphabet.}
	\Stmt lettreMaj(n: entier) \Gives~caractère			\RComment{donne la lettre majuscule de position donnée.}
	\Stmt lettreMin(n: entier) \Gives~caractère			\RComment{donne la lettre minuscule de position donnée.}
	\Stmt chaine(car: caractère) \Gives~chaine			\RComment{convertit le caractère en une chaine.}
	\Let varChaine \Gets~varCaractère					\RComment{idem}
	\Stmt chaine(n : entier) \Gives~chaine				\RComment{convertit un entier en une chaine.}
	\Stmt chaine(x : réel) \Gives~chaine				\RComment{convertit un réel en une chaine.}
	\Stmt nombre(ch : chaine) \Gives~réel				\RComment{convertit une chaine en un nombre.}
	\Empty
	\Stmt {\large // Manipulations}
	\Empty
	\Stmt longueur(ch : chaine) \Gives~entier			\RComment{donne la taille de la chaine.}
	\Stmt car(ch: chaine, n: entier) \Gives~caractère	\RComment{donne le caractère à une position donnée.}
	\Stmt sousChaine(ch: chaine, pos: entier, long: entier) \Gives~chaine \RComment{extrait une sous-chaine}
	\Stmt estDansChaine(ch: chaine, sous-chaine: chaine [ou caractère]) \Gives~entier 
	\Stmt 	\RComment{dit où commence une sous-chaine dans une chaine donnée (0 si pas trouvé)}
	\Stmt concat(ch1, ch2, \dots, chN: chaine) \Gives~chaine 	\RComment{concatène des chaines}
	\Let  ch \Gets~ch1 + ch2 + \dots + chN						\RComment{idem}
\end{Pseudocode}
