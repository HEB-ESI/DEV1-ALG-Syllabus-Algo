% ====================
\chapter{Les chaines}
% ====================

	\marginicon{objectif}
	Dans ce chapitre, 
	nous allons apprendre à manipuler du texte 
	en étudiant les différentes fonctions associées 
	aux variables de type chaine. 
	Ce sera aussi l'occasion de revoir quelques techniques algorithmiques 
	déjà acquises pour les nombres (alternatives, boucles) 
	afin de les consolider dans un contexte plus \og littéraire \fg.
	
\section{Introduction}
% ====================

	Très tôt dans ce cours,
	nous avons introduit le type \lda{chaine}
	mais nous n'en avons encore fait que des usages basiques,
	essentiellement des affichages.
	Il est temps d'aller plus loin et de les \emph{manipuler},
	c'est-à-dire d'examiner et de manipuler le contenu de chaines. 
	Mais procédons par étapes.
	
\section{Longueur}
%=================
	
	\marginicon{definition}
	La \textbf{longueur} d'une chaine est le nombre de caractères
	qu'elle contient. 
	Pour la connaitre on utilisera la notation \lda{long(uneChaine)}.
	Par exemple :
	\begin{itemize}
	\item \lda{long("Bonjour")} vaut 7.
	\item \lda{long("Une chaine")} vaut 10 (l'espace compte pour 1).
	\item \lda{long("A")} est 1.
	\item \lda{long("")} est égal à 0.
	\end{itemize}
	
\section{Chaine et caractère}
%============================

	Certains langages, comme Java,
	distinguent clairement le type chaine et le type caractère.
	Ainsi, en Java, le littéral \lda{'A'} représente un caractère
	qui est différent du littéral \lda{"A"} qui représente une chaine
	(composée d'un seul caractère).
	Cette distinction est essentiellement dictée par des détails
	d'implémentation; un caractère pouvant se représenter
	de façon plus économe qu'une chaine.

	D'autres langages, comme Python, n'ont pas de type caractère spécifique.
	D'autres encore, comme C, ont un type caractère mais pas de type chaine
	à proprement parler; ils sont vus comme des tableaux de caractères. 

	Ici, nous introduirons un type \lda{caractère}
	mais sans le distinguer clairement d'une chaine.
	Ainsi, un caractère pourra être utilisé comme une chaine
	et une chaine d'un seul caractère pourra être considéré comme
	un caractère. 
	En somme, \lda{caractère} n'est qu'un raccourci pour dire :
	\emph{une chaine de longueur 1}.
	
\section{Le contenu d'une chaine}
%=================================

	Pour pouvoir manipuler une chaine,
	il faut pouvoir accéder aux caractères qui la composent.
	Comme il s'agit d'une opération de base souvent utilisée,
	nous allons introduire une écriture compacte,
	empruntée aux tableaux : 
	\lda{chaine[i]} désigne le i\ieme{} caractère de la chaine
	(en commençant à 1).
	Par exemple :
	\begin{LDA}
		\Decl{texte}{chaine}
		\Let texte \Gets "Hello"
		\Write chaine[5]	\RComment Affiche "o"
		\Let chaine[2] \Gets "a" \RComment texte vaut "Hallo"; un caractère doit être remplacé par un seul caractère.
	\end{LDA}

	\paragraph{Exemple.}
	Écrivons un algorithme qui vérifie si un mot donné
	contient une lettre donnée.
	\begin{LDA}
		\Algo{contient}{\Par{mot}{chaine}, \Par{lettre}{caractère}}{booléen}
			\Decl{i}{entier}
			\Let i \Gets 1
			\While{i$\le$long(mot) ET mot[i]$\neq$lettre}
				\Let i \Gets i + 1
			\EndWhile
			\Return i$\le$long(mot)
		\EndAlgo
	\end{LDA}
	
\section{La concaténation}
%=================================

	Il est fréquent de devoir rassembler plusieurs chaines 
	pour former une seule chaine plus grande, 
	il s'agit de l'opération de \textbf{concaténation}.
	Introduisons également une notation compacte, le \lda{+}. 
	Par exemple : 
	\begin{LDA}
		\Let texte \Gets "al" + "go" + "rithmique"
		\RComment
		assigne la chaine "algortihmique" à la variable texte.
	\end{LDA}	

	\paragraph{Exemple.}
	Écrivons un algorithme qui inverse toutes les lettres d'un mot.
	Ainsi, "algo" deviendra "ogla".
	\begin{LDA}
		\Algo{miroir}{\Par{mot}{chaine}}{chaine}
			\Decl{miroir}{chaine}
			\Let miroir \Gets ""
			\For{i}{1}{long(mot)}
				\Let miroir \Gets mot[i] + miroir
			\EndFor
			\Return miroir
		\EndAlgo
	\end{LDA}
	
\section{Manipuler les caractères}
%================================

	Les algorithmes qui suivent pourraient être écrits
	avec ce que nous avons déjà introduit mais ce serait long
	et fastidieux. 
	Nous allons supposer qu'ils existent et on pourra les utiliser
	si nécessaire.
	
	\begin{description}
	\item[\lda{estLettre(car: caractère)\Gives~booléen}]
		Cette fonction indique si un caractère est une lettre. 
		Par exemple elle retourne vrai pour 'a', 'e', 'G', 'K', 
		mais faux pour '4', '\$', '@'\dots %$ 
	\item[\lda{estMinuscule(car: caractère)\Gives~booléen}]	
		Permet de savoir si le caractère est une lettre minuscule.
	\item[\lda{estMajuscule(car: caractère)\Gives~booléen}]	
		Permet de savoir si le caractère est une lettre minuscule.
	\item[\lda{estChiffre(car: caractère)\Gives~booléen}]	
		Permet de savoir si un caractère est un chiffre. 
		Elle retourne vrai uniquement pour les dix caractères 
		'0', '1', '2', '3', '4', '5', '6', '7', '8' et '9' 
		et faux dans tous les autres cas.
	\item[\lda{majuscule(texte: chaine)\Gives~chaine}]
		Retourne une chaine où toutes les lettres du texte
		ont été converties en majuscule.
	\item[\lda{minuscule(texte: chaine)\Gives~chaine}]
		Retourne une chaine où toutes les lettres du texte
		ont été converties en minuscule.
	\end{description}
	
\section{L'alphabet}
%===================

	Il peut aussi être pratique de connaitre 
	la position d'une lettre dans l'alphabet. 
	Ceci se fera à l'aide de la fonction :

	\begin{description}
	\item[\lda{numLettre(car: caractère)\Gives~entier}]
		Retourne toujours un entier entre 1 et 26. 
		Par exemple \lda{numLettre('E')} donnera 5, 
		ainsi que \lda{numLettre('e')}, 
		cette fonction traite donc de la même manière 
		les majuscules et les minuscules. 
		\lda{numLettre} retournera aussi 5 pour les caractères 'é', 'è', 'ê', 'ë'\dots). 
		Attention, il est interdit d'utiliser cette fonction 
		si le caractère n'est pas une lettre !
	\end{description}
	
	Il peut être utile d'avoir un outil qui fait l'opération inverse, 
	à savoir associer la lettre de l'alphabet correspondant à une position donnée. 
	Pour cela, nous aurons : 

	\begin{description}
	\item[\lda{lettreMaj(n: entier)\Gives~caractère}]
		Retourne la forme majuscule de la n\ieme{} lettre de l'alphabet 
		(où \textit{n} sera obligatoirement compris entre 1 et 26). 
		Par exemple, \lda{lettreMaj(13)} retourne 'M'.
	\item[\lda{lettreMin(n: entier)\Gives~caractère}]
		Idem pour les minuscules.
	\end{description}
	
\section{Chaine et nombre}
%=========================

	Si une chaine contient un nombre,
	on doit pouvoir le convertir, et inversément.

	\begin{description}
	\item[\lda{chaine(n: réel)\Gives~chaine}]
		Transforme un nombre en chaine.
		Ex: \lda{chaine(42)} retourne la chaine "42"
		et \lda{chaine (3,14)} donnera "3,14". 
	\item[\lda{nombre(ch : chaine) \Gives~réel}]
		Transforme une chaine contenant des caractères numériques 
		en nombre.
		Ainsi, \lda{nombre("3,14")} retournera 3,14. 
		On supposera encore que si la chaine \textit{ch} ne représente pas 
		un nombre de façon correcte, la fonction retournera 0.
	\end{description}
	
\section{Extraction de sous-chaines}
% ==================================

	\begin{description}
	\item[\lda{sousChaine(ch: chaine, pos: entier, long: entier) \Gives~chaine}]
		Permet d'extraire une portion 
		d'une certaine longueur d'une chaine donnée, 
		et ceci à partir d'une position donnée. 
	\end{description}

	Il faudra aussi être très vigilant 
	pour une utilisation correcte: 
	\lda{pos} doit être compris entre 1 et la taille de la chaine, 
	et la valeur de \lda{long} doit être telle 
	qu'on ne déborde pas de la chaine ! 

	Par exemple, \lda{sousChaine("algorithmique", 5, 3)} 
	donne "rit".

	Cette fonction est très utile pour sélectionner 
	des portions d'une chaine 
	contenant des informations codées sous un certain format. 
	Prenons par exemple une date stockée dans une chaine \textit{ch} de format 
	"JJ/MM/AAAA" (de longueur 10). 
	Pour extraire le jour de cette date, 
	on écrira \lda{sousChaine(ch, 1, 2)} ; 
	le mois s'obtiendra avec \lda{sousChaine(ch, 4, 2)} 
	et l'année avec \lda{sousChaine(ch, 7, 4)}. 
	Attention, les 3 résultats obtenus sont des chaines de chiffres 
	et non des nombres mais il est possible de les convertir
	via la fonction \lda{nombre} !

\section{Recherche de sous-chaine}
% ==================================
	
	\begin{description}
	\item[\lda{position(ch: chaine, sous-chaine: chaine)\Gives~entier}]	
		Permet de savoir 
		si une sous-chaine donnée 
		est présente dans une chaine donnée. 
		Elle permet d'éviter d'écrire 
		le code correspondant à une recherche. 
	\end{description}
	
	La valeur de l'entier renvoyé est la position 
	où commence la sous-chaine recherchée. 
	Par exemple, 
	\lda{position("algorithmique", "mi")} retournera 9. 
	Si la sous-chaine ne s'y trouve pas, 
	la fonction retourne 0. 
	On peut admettre ici d'écrire un caractère à la place de la sous-chaine. 
	Par exemple, 
	\lda{position("algorithmique", "m")} retournera également 9. 

\section{Exercices}
%==================

	\begin{Exercice}{Calcul de fraction}
		Écrire un algorithme 
		qui reçoit une fraction sous forme de chaine, 
		et retourne la valeur numérique de celle-ci. 
		Par exemple, si la fraction donnée est "5/8", 
		l'algorithme renverra 0,625. 
		On peut considérer que la fraction donnée est correcte, 
		elle est composée de 2 entiers séparés 
		par le caractère de division '/'.
		
		\paragraph{Solution.}
		Voici une solution possible.
		\begin{LDA}
			\Algo{calculerFraction}{\Par{fraction}{chaine}}{réel}
				\Decl{posFraction, numérateur, dénominateur}{entiers}
				\Let posFraction \Gets position(fraction,"/")
				\Let numérateur \Gets nombre(souschaine(fraction,1,posFraction-1))
				\Let dénominateur \Gets nombre(souschaine(fraction,posFraction+1,long(fraction)-posFraction))
				\Return numérateur / dénominateur
			\EndAlgo
		\end{LDA}
	\end{Exercice}
	
	\begin{Exercice}{Conversion de nom}
		Écrire un algorithme 
		qui reçoit le nom complet d'une personne 
		dans une chaine sous la forme "nom, prénom" 
		et la renvoie au format "prénom nom" 
		(sans virgule séparatrice). 
		Exemple : "De Groote, Jan" deviendra "Jan De Groote".	
	\end{Exercice}
	
	\begin{Exercice}{Gauche et droite}
		Écrire un algorithme \lda{gauche} 
		et un \lda{droite}
		qui retourne la chaine formée respectivement 
		des n premiers et des n derniers caractères d'une chaine donnée.	
	\end{Exercice}
	
	\begin{Exercice}{Grammaire}
		Écrire un algorithme 
		qui met un mot en « ou » au pluriel. 
		Pour rappel, 
		un mot en « ou » prend un « s » à l'exception des 7 mots 
		bijou, caillou, chou, genou, hibou, joujou et pou qui prennent 
		un « x » au pluriel. 
		Exemple : un clou, des clous, un hibou, des hiboux. 
		Si le mot soumis à l'algorithme n'est pas un mot en « ou », 
		un message adéquat sera affiché.
	\end{Exercice}

	\begin{Exercice}{Normaliser une chaine}
		Écrire un module qui reçoit une chaine et retourne une autre chaine,
		version normalisée de la première.
		Par normalisée, on entend~:~enlever tout ce qui n'est pas une lettre 
		et tout mettre en majuscule.
		\\Exemple~:~"Le <COBOL>, c'est la santé~!" devient "LECOBOLCESTLASANTE".
	\end{Exercice}
		
	\begin{Exercice}{Les palindromes}
		Cet exercice a déjà été réalisé pour des entiers, 
		en voici 2 versions « chaine » ! 
		\begin{enumerate}[label=\alph*)]
		\item 
			Écrire un algorithme qui vérifie 
			si un mot donné sous forme de chaine 
			constitue un palindrome 
			(comme par exemple "kayak", "radar" ou "saippuakivikauppias" 
			(marchand de savon en finnois)
		\item
			Écrire un algorithme qui vérifie 
			si une phrase donnée sous forme de chaine constitue un palindrome 
			(comme par exemple "Esope reste ici et se repose" 
			ou "Tu l'as trop écrasé, César, ce Port-Salut !"). 
			Dans cette seconde version, 
			on fait abstraction des majuscules/minuscules 
			et on néglige les espaces et tout signe de ponctuation.
		\end{enumerate}
	\end{Exercice}
	
	\begin{Exercice}{Le chiffre de César}
		\label{ex:cesar}
		Depuis l'antiquité, les hommes politiques, les militaires, 
		les hommes d'affaires cherchent à garder secret les messages
		importants qu'ils doivent envoyer.
		L'empereur César utilisait une technique (on dit un \emph{chiffrement})
		qui porte à présent son nom~:
		remplacer chaque lettre du message par la lettre qui se situe 
		$k$ positions plus loin dans l'alphabet (cycliquement).
		
		Exemple~:~si $k$ vaut $2$, 
		alors le texte clair "CESAR" devient "EGUCT" lorsqu'il est chiffré 
		et le texte "ZUT" devient "BWV".
		
		Bien sûr, il faut que l'expéditeur du message et le récepteur
		se soient mis d'accord sur la valeur de $k$.
		
		On vous demande d'écrire un algorithme qui reçoit une chaine ne contenant
		que des lettres majuscules ainsi que la valeur de $k$ et qui retourne
		la version chiffrée du message.
	
		On vous demande également d'écrire l'algorithme de déchiffrement.
		Cet algorithme reçoit un message chiffré et la valeur de $k$ qui a été
		utilisée pour le chiffrer et retourne le message en clair.
		Attention~! Ce second algorithme est \textbf{très simple}.
	\end{Exercice}

	\begin{Exercice}{Remplacement de sous-chaines}
		Écrire un algorithme 
		qui remplace dans une chaine donnée 
		toutes les sous-chaines ch1 par la sous-chaine ch2. 
		Attention, 
		cet exercice est plus coriace qu'il n'y parait à première vue\dots~
		Assurez-vous que votre code n'engendre pas de boucle infinie. 
	\end{Exercice}	
	
	
