%===========================================
\chapter{Une question de choix}
%===========================================

	Vous avez déjà eu l’occasion d’aborder les alternatives
	lors de votre initiation aux algorithmes sur le site
	\url{code.org}.
	Par exemple, vous avez indiqué au zombie quelque chose comme~:
	\og{}S’il existe un chemin à gauche alors tourner à gauche\fg{}.
	
	Les \textbf{alternatives}\index{alternatives} 
	permettent de n’exécuter des instructions
	que si une certaine \emph{condition} est vérifiée.
	Avec le zombie, vous testiez son environnement~;
	dans nos algorithmes, vous allez tester les données.
	
	Les algorithmes vus jusqu’à présent ne proposent
	qu’un seul \og{}chemin\fg{}, une seule \og{}histoire\fg{}.
	À chaque exécution de l’algorithme,
	les mêmes instructions s’exécutent dans le même ordres.
	Les alternatives permettent de créer des histoires différentes,
	d’adapter les instructions aux valeurs concrètes des données.
	Procédons par étapes.

%===========================================
\section{Le si}
\index{si}
%===========================================
		
	Il existe des situations où
	des instructions ne doivent pas toujours être exécutées
	et un test va nous permettre de le savoir.
	
	\paragraph{Exemple.}
	Supposons que la variable \lda{nb}
	contienne un nombre positif ou négatif,
	on ne sait pas.
	Et supposons qu’on veuille le rendre positif.
	On peut tester son signe.
	S’il est négatif on peut l’inverser.
	Par contre, s’il est positif,
	il n’y a rien à faire.
	Voici comment on peut l’écrire,
	graphiquement%
	\footnote{%
		Ce graphique, appelé 
		\emph{organigramme} ou encore \emph{ordinogramme}
		permet de représenter un algorithme
		de façon plus visuelle.
		cf. \url{http://fr.wikipedia.org/wiki/Organigramme_de_programmation}.
	} et en LDA~:
	
	\qquad
	\begin{minipage}{6cm}
		\begin{tikzpicture}[node distance = 1.5cm, auto]
			\sffamily				
			\node[draw, diamond] (Test) {nb<0 ?};
			\node[draw, below of=Test] (Instr) {nb \Gets -nb};
			\node[right of=Instr] (Empty) {};		          
			\node[draw, circle, below of=Instr] (End) {};		          
			\draw[->] (Test) -- (Instr) node[near start] {Vrai};
			\draw[->] (Instr) -- (End);
			\draw[->] (Test) to[out=0,in=0] node[near start] {Faux}  (End);
		\end{tikzpicture}	
	\end{minipage}
	\quad
	\begin{minipage}{6cm}
		\begin{LDA}[1]
			\If{nb<0}
				\Let nb \Gets -nb
			\EndIf
		\end{LDA}
	\end{minipage}

	Traçons-le dans deux cas différents pour bien illustrer 
	son déroulement.
	
	\begin{center}
	\begin{tabular}{|>{\centering\arraybackslash}m{1cm}
					|*{2}{>{\centering\arraybackslash}m{1cm}}|}
		\hline
			\verb_#_  & nb & test \\			
		\hline
			  & -3                   & {}   \\
			1 & {\color{gray}$\mid$} & vrai \\
			2 & 3                    & {}   \\
		\hline
	\end{tabular}
	\qquad
	\begin{tabular}{|>{\centering\arraybackslash}m{1cm}
					|*{2}{>{\centering\arraybackslash}m{1cm}}|}
		\hline
			\verb_#_  & nb & test \\			
		\hline
			  & 3                    & {}   \\
			1 & {\color{gray}$\mid$} & faux \\
		\hline
	\end{tabular}
	\end{center}
	
	La condition peut être n’importe quelle expression (calcul)
	dont le résultat est un booléen (vrai ou faux).

	\marginicon{attention}
	Attention~!
	Vous faites parfois la confusion.
	Un \og{}si\fg{} n’est pas une règle que l’ordinateur
	doit apprendre et exécuter à chaque fois que l’occasion se présente.
	La condition n’est testée que lorsqu’on arrive à cet endroit
	de l’algorithme.

	\begin{Exercice}{Compréhension}
		Tracez cet algorithme et donnez la valeur de retour.
		\begin{LDA}
		\Algo{exerciceA}{a, b~: entiers}{entier}
			\Decl{c}{entier}
			\Let c \Gets 2 * a
			\If{c > b}
				\Let c \Gets c - b
			\EndIf
			\Return c
		\EndAlgo
		\end{LDA}		
		\begin{multicols}{2}
		\begin{itemize}
		\item \lda{exerciceA(2, 5)} = \_\_\_
		\item \lda{exerciceA(4, 1)} = \_\_\_
		\end{itemize}
		\end{multicols}	
	\end{Exercice}

	\begin{Exercice}{Simplification d’algorithmes}
		Voici quelques extraits d’algorithmes corrects 
		du point de vue de la syntaxe 
		mais contenant des lourdeurs d’écriture.
		Simplifiez-les.
		
		\begin{LDA}
		\If{ok = vrai}
			\Write nombre
		\EndIf
		\end{LDA}
	
		\begin{LDA}
		\If{ok = faux}
			\Write nombre
		\EndIf
		\end{LDA}
	
		\begin{LDA}
		\If{ok = vrai OU ok = faux}
			\Write nombre
		\EndIf
		\end{LDA}
	
		\begin{LDA}
		\If{ok = vrai ET ok = faux}
			\Write nombre
		\EndIf
		\end{LDA}
	\end{Exercice}

\pagebreak		
%===============================
\section{Le si-sinon}
\index{si-sinon}
%===============================
	
	La construction \lda{si-sinon}
	permet d’exécuter certaines instructions
	ou d’autres en fonction d’un test.
	Pour illustrer cette instruction,
	nous allons nous pencher sur un grand classique,
	la recherche de maximum.
	
	\paragraph{Exemple.}
	Supposons qu’on veuille déterminer
	le maximum de deux nombres,
	c’est-à-dire la plus grande des deux valeurs.
	Dans la solution, il y a deux chemins possibles.
	Le maximum devra pendre la valeur du premier nombre
	ou du second selon que le premier est plus grand 
	que le second ou pas.
	
	\qquad
	\begin{minipage}{6cm}
		\begin{tikzpicture}[node distance = 1.5cm, auto]
			\sffamily				
			\node[draw, diamond] (Test) {nb1>nb2 ?};
			\node[below of=Test] (Empty) {};
			\node[draw,left  of=Empty] (If) {max \Gets nb1};		          
			\node[draw,right  of=Empty] (Else) {max \Gets nb2};		          
			\node[draw, circle, below of=Empty] (End) {};
			\draw[->] (Test) -| (If) node[left, midway] {Vrai};
			\draw[->] (If) |- (End);
			\draw[->] (Test) -| (Else) node[right, midway] {Faux};
			\draw[->] (Else) |- (End);
		\end{tikzpicture}	
	\end{minipage}
	\quad
	\begin{minipage}{6cm}
		\begin{LDA}[1]
			\If{nb1>nb2}
				\Let max \Gets nb1
			\Else
				\Let max \Gets nb2
			\EndIf
		\end{LDA}
	\end{minipage}

	Traçons-le dans différentes situations.
	
	\begin{tabular}{|>{\centering\arraybackslash}m{6mm}
					|*{4}{>{\centering\arraybackslash}m{1cm}}|}
		\hline
			\verb_#_  & nb1 & nb2 & max & test \\			
		\hline
			  & 3 & 2 & indéfini & {} \\
			1 & {\color{gray}$\mid$} & {\color{gray}$\mid$} & indéfini & vrai \\
			2 & {\color{gray}$\mid$} & {\color{gray}$\mid$} & 3        & {} \\
		\hline
	\end{tabular}
	\quad
	\begin{tabular}{|>{\centering\arraybackslash}m{6mm}
					|*{4}{>{\centering\arraybackslash}m{1cm}}|}
		\hline
			\verb_#_  & nb1 & nb2 & max & test \\			
		\hline
			  & 4 & 42 & indéfini & {} \\
			1 & {\color{gray}$\mid$} & {\color{gray}$\mid$} & indéfini & faux \\
			4 & {\color{gray}$\mid$} & {\color{gray}$\mid$} & 42        & {} \\
		\hline
	\end{tabular}
	
	Le cas où les deux nombres sont égaux est également géré.
	
	\begin{tabular}{|>{\centering\arraybackslash}m{1cm}
					|*{4}{>{\centering\arraybackslash}m{1cm}}|}
		\hline
			\verb_#_  & nb1 & nb2 & max & test \\			
		\hline
			  & 4 & 4 & indéfini & {} \\
			1 & {\color{gray}$\mid$} & {\color{gray}$\mid$} & indéfini & faux \\
			4 & {\color{gray}$\mid$} & {\color{gray}$\mid$} & 4        & {} \\
		\hline
	\end{tabular}
	
	\begin{Exercice}{Compréhension}
		Tracez ces algorithmes et donnez la valeur de retour.
	
		\begin{LDA}
		\Algo{exerciceB}{b, a : entiers}{entier}
			\Decl{c}{entier}
			\If{a > b}
				\Let c \Gets a DIV b
			\Else
				\Let c \Gets b MOD a	
			\EndIf
			\Return c
		\EndAlgo
		\end{LDA}
	
		\begin{multicols}{2}
		\begin{itemize}
		\item \lda{exerciceB(2, 3)} = \_\_\_
		\item \lda{exerciceB(4, 1)} = \_\_\_
		\end{itemize}
		\end{multicols}
	
		\begin{LDA}
		\Algo{exerciceC}{x1, x2~: entiers}{entier}
			\Decl{ok}{booléen}
			\Let ok \Gets x1 > x2
			\If{ok}
				\Let ok \Gets ok ET x1 = 4
			\Else
				\Let ok \Gets ok OU x2 = 3
			\EndIf
			\If{ok}
				\Let x1 \Gets x1 * 1000
			\EndIf
			\Return x1 + x2
		\EndAlgo
		\end{LDA}
		
		\medskip
		\begin{multicols}{2}
		\begin{itemize}
		\item \lda{exerciceC(2, 3)} = \_\_\_
		\item \lda{exerciceC(4, 1)} = \_\_\_	
		\end{itemize}
		\end{multicols}	
		
	\end{Exercice}	
	
	\begin{Exercice}{Simplification d’algorithmes}
		Voici quelques extraits d’algorithmes corrects du point de vue de la
		syntaxe mais contenant des lignes inutiles ou des lourdeurs d’écriture.
		Remplacer chacune de ces portions d’algorithme par un minimum
		d’instructions qui auront un effet équivalent.
	
		\begin{minipage}[t]{7cm}
			\begin{LDA}
			\If{condition}
				\Let ok \Gets vrai
			\Else
				\Let ok \Gets faux
			\EndIf
			\end{LDA}
		\end{minipage}
		\quad
		\begin{minipage}[t]{7cm}		
			\begin{LDA}
			\If{a $>$ b}
				\Let ok \Gets faux
			\Else
				\If{a $\leq$ b}
					\Let ok \Gets vrai
				\EndIf
			\EndIf
			\end{LDA}
		\end{minipage}
	\end{Exercice}

	\begin{Exercice}{Maximum de 2 nombres}
		\marginicon{java}
		Écrire un algorithme qui, étant donné deux nombres quelconques,
		recherche et retourne le plus grand des deux. Attention~! On ne veut
		pas savoir si c’est le premier ou le deuxième qui est
		le plus grand mais bien quelle est cette plus grande valeur. Le
		problème est donc bien défini même si les deux nombres sont
		identiques.
		
		\textbf{Solution.}
		Une solution complète est disponible dans la fiche \vref{fiche:max2nb}.
	\end{Exercice}
	
	\begin{Exercice}{Calcul de salaire}
		Dans une entreprise, 
		une retenue spéciale de 15\% est pratiquée 
		sur la partie du salaire mensuel qui dépasse 1200~\texteuro. 
		Écrire un algorithme qui calcule le salaire net à partir du salaire brut. 
		En quoi l’utilisation de constantes convient-elle pour améliorer cet algorithme~?
	\end{Exercice}

	\begin{Exercice}{Fonction de Syracuse}
		\marginicon{java}
		Écrire un algorithme qui, étant donné un entier $n$ quelconque,
		retourne le résultat de la fonction
		$f(n)=
			\left\{
			\begin{array}{rl}
				n/2 & si \ n \ est\ pair\\
				3n+1 & si \ n \ est \ impair
			\end{array}
			\right.$
	\end{Exercice}

	\begin{Exercice}{Tarif réduit ou pas}
		Dans une salle de cinéma,
		le tarif plein pour une place est de 8\texteuro{}.
		Les personnes ayant droit au tarif réduit payent 7\texteuro{}.
		Écrire un algorithme qui reçoit un booléen
		indiquant si la personne peut bénéficier du tarif réduit
		et qui retourne le prix à payer.
	\end{Exercice}

%============================
\section{Le si-sinon-si}
\index{si-sinon-si}
%============================

	Avec cette construction,
	on peut indiquer à un endroit de l’algorithme
	plus que deux chemins possibles.
	Partons à nouveau d’un exemple pour illustrer cette instruction.
	
	\paragraph{Exemple.}
	On voudrait mettre dans la chaine \lda{signe}
	la valeur \lda{"positif"}, \lda{"négatif"}
	ou \lda{"nul"}
	selon qu’un nombre donné est positif, négatif ou nul.
	
	Ici, lorsqu’on va examiner le nombre, 
	trois chemins vont s’offrir à nous.

	\begin{minipage}{9cm}
		\begin{tikzpicture}[auto]
			\sffamily				
			\node[draw, diamond] (Test) {\ \ };
			\node[draw,below=1cm of Test] (nul) {signe \Gets "nul"};
			\node[draw,left=0.5cm  of nul] (pos) {signe \Gets "positif"};		          
			\node[draw,right=0.5cm  of nul] (neg) {signe \Gets "négatif"};		          
			\node[draw, circle, below=1cm of nul] (End) {};
			\draw[->] (Test) -| (pos) node[left, near end] {nb>0};
			\draw[->] (Test) -- (nul) node[midway] {nb=0};
			\draw[->] (Test) -| (neg) node[right, near end] {nb<0};
			\draw[->] (pos) |- (End);
			\draw[->] (nul) -- (End);
			\draw[->] (neg) |- (End);
		\end{tikzpicture}	
	\end{minipage}
	\quad
	\begin{minipage}{4cm}
		\begin{LDA}[1]
			\If{nb>0}
				\Let signe \Gets "positif"
			\ElsIf{nb=0}
				\Let signe \Gets "nul"
			\Else
				\Let signe \Gets "négatif"
			\EndIf
		\end{LDA}
	\end{minipage}
	
	Traçons-le
	
	\begin{tabular}{|*{2}{>{\centering\arraybackslash}m{4mm}}
					 *{2}{>{\centering\arraybackslash}m{9mm}}|}
		\hline
			\verb_#_  & nb & signe & test \\			
		\hline
			  & 2                    & indéfini             & {}   \\
			1 & {\color{gray}$\mid$} & {\color{gray}$\mid$} & vrai \\
			2 & {\color{gray}$\mid$} & "positif"            & {}   \\
		\hline
	\end{tabular}
	\quad
	\begin{tabular}{|*{2}{>{\centering\arraybackslash}m{4mm}}
					 *{2}{>{\centering\arraybackslash}m{9mm}}|}
		\hline
			\verb_#_  & nb & signe & test \\			
		\hline
			  & 0                    & indéfini             & {}   \\
			1 & {\color{gray}$\mid$} & {\color{gray}$\mid$} & faux \\
			3 & {\color{gray}$\mid$} & {\color{gray}$\mid$} & vrai \\
			4 & {\color{gray}$\mid$} & "nul"                & {}   \\
		\hline
	\end{tabular}
	\quad
	\begin{tabular}{|*{2}{>{\centering\arraybackslash}m{4mm}}
					 *{2}{>{\centering\arraybackslash}m{9mm}}|}
		\hline
			\verb_#_  & nb & signe & test \\			
		\hline
			  & 0                    & indéfini             & {}   \\
			1 & {\color{gray}$\mid$} & {\color{gray}$\mid$} & faux \\
			3 & {\color{gray}$\mid$} & {\color{gray}$\mid$} & faux \\
			6 & {\color{gray}$\mid$} & "négatif"            & {}   \\
		\hline
	\end{tabular}
	
	\paragraph{Remarques.}
	\begin{itemize}
	\item
		Pour le dernier cas, on se contente
		d’un \lda{\K{sinon}} sans indiquer la condition~;
		ce serait inutile, elle serait toujours vraie.
	\item
		Le \lda{\K{si}} et le \lda{\K{si-sinon}} 
		peuvent être vus comme des cas particuliers du 
		\lda{\K{si-sinon-si}}.
	\item
		On pourrait écrire la même chose 
		avec des \lda{\K{si-sinon}} imbriqués
		mais le \lda{\K{si-sinon-si}} est plus lisible.
		
		%\begin{wrong}
		\begin{LDA}
			\If{nb>0}
				\Let signe \Gets "positif"
			\Else
				\If{nb=0}
					\Let signe \Gets "nul"
				\Else
					\Let signe \Gets "négatif"
				\EndIf
			\EndIf
		\end{LDA}
		%\end{wrong}
	\item
		Lorsqu’une condition est testée,
		on sait que toutes celles au-dessus
		se sont avérées fausses.
		Cela permet parfois de simplifier la condition.

		\textbf{Exemple.}
		Supposons que le prix unitaire d’un produit (\lda{prixUnitaire})
		dépende de la quantité achetée (\lda{quantité}). 
		En dessous de 10 unités, on le paie 10\texteuro{} l’unité.
		De 10 à 99 unités, on le paie 8\texteuro{} l’unité.
		À partir de 100 unités, on paie 6\texteuro{} l’unité.
		
		\begin{LDA}
			\If{quantité<10}
				\Let prixUnitaire \Gets 10
			\ElsIf{quantité<100} \Comment{On sait que ce n’est pas <10; inutile de le tester}
				\Let prixUnitaire \Gets 8
			\Else
				\Let prixUnitaire \Gets 6
			\EndIf
		\end{LDA}
	\end{itemize}
		
	\begin{Exercice}{Maximum de 3 nombres}
		\marginicon{java}
		Écrire un algorithme qui, étant donné trois nombres quelconques,
		recherche et retourne le plus grand des trois.
	\end{Exercice}
	
	\begin{Exercice}{Le signe}
		\marginicon{java}
		Écrire un algorithme qui \textbf{affiche} un message indiquant
		si un entier est strictement négatif, nul ou strictement
		positif.
	\end{Exercice}

	\begin{Exercice}{Le type de triangle}
		Écrire un algorithme qui indique si un triangle
		dont on donne les longueurs de ces 3 cotés est~:
		équilatéral (tous égaux), isocèle (2 égaux)
		ou quelconque.
	\end{Exercice}

	\begin{Exercice}{Dés identiques}
		Écrire un algorithme qui lance trois dés
		et indique si on a obtenu 3 dés de valeur identique,
		2 ou aucun.
	\end{Exercice}

	\begin{Exercice}{Grade}
		\marginicon{java}
		Écrire un algorithme qui retourne le grade d’un étudiant 
		suivant la moyenne qu’il a obtenue.
		
		Un étudiant ayant obtenu 
		\begin{itemize}
			\item moins de 50\% n’a pas réussi~;
			\item de 50\% inclus à 60\% exclu a réussi~;
			\item de 60\% inclus à 70\% exclu a une satisfaction~;
			\item de 70\% inclus à 80\% exclu a une distinction~;
			\item de 80\% inclus à 90\% exclu a une grande distinction~;
			\item de 90\% inclus à 100\% inclus a la plus grande distinction.
		\end{itemize}
	\end{Exercice}

%\clearpage
%=====================
\section{Le selon-que}
\index{selon-que}
%=====================

	Cette nouvelle instruction permet d’écrire plus lisiblement
	\emph{certains} \lda{\K{si-sinon-si}},
	plus précisément quand le choix d’une branche dépend
	de la valeur précise d’une variable
	(ou d’une expression).

	\paragraph{Exemple.}
	Imaginons qu’une variable (\lda{numéroJour}) 
	contienne un numéro de jour de la semaine
	et qu’on veuille mettre dans une variable (\lda{nomJour})
	le nom du jour correspondant 
	("lundi" pour 1, "mardi" pour 2\dots)
	
	On peut écrire une solution avec un \lda{\K{si-sinon-si}}
	mais le \lda{\K{selon-que}} est plus lisible.

	\begin{minipage}{6cm}
		\begin{LDA}
			\Switch{numéroJour}
			\Case{1} nomJour \Gets "lundi"
			\Case{2} nomJour \Gets "mardi"
			\Case{3} nomJour \Gets "mercredi"
			\Case{4} nomJour \Gets "jeudi"
			\Case{5} nomJour \Gets "vendredi"
			\Case{6} nomJour \Gets "samedi"
			\Case{7} nomJour \Gets "dimanche"
			\EndSwitch
		\end{LDA}
	\end{minipage}

	remplace avantageusement

	\begin{wrong}
		\begin{LDA}
			\If{numéroJour=1}
				\Let nomJour \Gets "lundi"
			\ElsIf{numéroJour=2}
				\Let nomJour \Gets "mardi"
			\ElsIf{numéroJour=3}
				\Let nomJour \Gets "mercredi"
			\ElsIf{numéroJour=4}
				\Let nomJour \Gets "jeudi"
			\ElsIf{numéroJour=5}
				\Let nomJour \Gets "vendredi"
			\ElsIf{numéroJour=6}
				\Let nomJour \Gets "samedi"
			\Else
				\Let nomJour \Gets "dimanche"
			\EndIf
		\end{LDA}
	\end{wrong}
		
	\paragraph{Remarques.}
	\begin{itemize}
	\item
		On peut spécifier plusieurs valeurs pour un cas donné.
	\item
		On peut mettre un cas \lda{\K{défaut}}
		qui sera exécuté si la valeur n’est pas reprise par ailleurs.
	\end{itemize}
	
	La syntaxe générale est~:
	
	\begin{LDA}
		\Switch{expression}
			\Case{liste${}_1$ de valeurs séparées par des virgules }
				\Stmt Instructions
			\Case{liste${}_2$ de valeurs séparées par des virgules }
				\Stmt Instructions
			\Empty \dots
			\Case{liste${}_k$ de valeurs séparées par des virgules }
				\Stmt Instructions
			\Case{\K{autres }}
				\Stmt Instructions
		\EndSwitch
	\end{LDA}
	
	\begin{Exercice}{Numéro du jour}
		\marginicon{java}
		Écrire un algorithme qui retourne le numéro du jour de la semaine
		reçu en paramètre (1 pour "lundi", 2 pour "mardi"\dots).
	\end{Exercice}

	\begin{Exercice}{Tirer une carte}
		\marginicon{java}
		Écrire un algorithme qui affiche l’intitulé d’une carte
		tirée au hasard dans un paquet de 52 cartes.
		Par exemple, "As de cœur", "3 de pique", "Valet de carreau"
		ou encore "Roi de trèfle".
		
		\textbf{Remarque}. Il est plus facile de déterminer séparément
		chacune des deux caractéristiques de la carte : couleur et valeur.
	\end{Exercice}
	
	\begin{Exercice}{Nombre de jours dans un mois}
		Écrire un algorithme qui retourne le nombre de jours dans un mois. 
		Le mois est lu sous forme d’un entier (1 pour janvier\dots).
		On considère dans cet exercice que le mois de février
		comprend toujours 28 jours.
	\end{Exercice}
		
%==============================
\section{Exercices de synthèse}
%==============================

	Dans les exercices qui suivent,
	à vous de déterminer si une instruction de choix est nécessaire
	et laquelle est la plus adaptée.

	\begin{Exercice}{Réussir DEV1}
		\marginicon{java}
		Pour réussir l’UE (unité d’enseignement) DEV1,
		il faut que la cote attribuée à cette UE 
		soit supérieure ou égale à 50\%.
		Cette cote tient compte de votre examen intégré
		et de vos interrogations.
		Écrire un algorithme 
		qui reçoit la cote finale (sur 100)
		d’un étudiant pour l’UE DEV1
		et qui indique si l’étudiant a réussi cette UE.
	\end{Exercice}		

	\begin{Exercice}{Réussir GEN1}
		\marginicon{java}
		\label{algo:réussirGEN1}
		l'UE (Unité d'enseignement) GEN1 est composée de trois AA (activité d’apprentissage) :
		Mathématique, Communication anglophone et Comptabilité%
		\footnote{%
			Sans parler de Méthodologie qui ne donne pas lieu à une évaluation.
		}.
		Pour réussir cette unité d’enseignement,
		il faut que la cote attribuée à chaque AA soit supérieure ou égale à 50\%.
		Si c'est le cas, la cote attribuée à l'UE est une moyenne \textbf{pondérée}
		des trois cotes d'AA 
		(avec la pondération 6 pour Mathématique et 2 pour les autres AA).
		
		Écrire un algorithme qui reçoit les 3 cotes (sur 20) d’AA d’un étudiant
		pour l’UE GEN1 et qui \textbf{affiche} un message
		indiquant si l’étudiant a réussi ou pas cette UE.
		S’il a réussi, l’algorithme affiche également la cote d'UE (sur 20).
	\end{Exercice}		
	
	\begin{Exercice}{La fourchette}
		\marginicon{java}
		Écrire un algorithme qui, étant donné trois nombres, 
		retourne vrai si le premier des trois 
		appartient à l’intervalle donné par le plus petit et le plus grand 
		des deux autres (bornes exclues) et faux sinon. 
		Qu’est-ce qui change si on inclut les bornes~?
	\end{Exercice}

	\begin{Exercice}{Le prix des photocopies}
		\marginicon{java}
		Un magasin de photocopies facture 0,10 \texteuro{} 
		les dix premières photocopies, 
		0,09 \texteuro{} les vingt suivantes 
		et 0,08 \texteuro{} au-delà. 
		Écrivez un algorithme 
		qui reçoit le nombre de photocopies effectuées 
		et qui affiche la facture correspondante.
	\end{Exercice}

	\begin{Exercice}{Le stationnement alternatif}
		\marginicon{java}
		Dans une rue où se pratique le stationnement alternatif, 
		du 1 au 15 du mois, on se gare du côté des maisons ayant un numéro impair, 
		et le reste du mois, on se gare de l’autre côté. 
		Écrire un algorithme qui, sur base de la date du jour et du numéro de maison
		devant laquelle vous vous êtes arrêté, 
		retourne vrai si vous êtes bien stationné et faux sinon.
	\end{Exercice}

	
