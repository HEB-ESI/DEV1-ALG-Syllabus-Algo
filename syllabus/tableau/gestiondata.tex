%==========================================
\chapter{Gérer les données dans un tableau}
%==========================================

	\begin{TODO}
		Voir ce qui va devenir des fiches.
	\end{TODO}
	
	Une utilisation courante des tableaux
	est le stockage des données changeantes.
	Lors de l'évolution de l'algorithme
	des données sont ajoutées, recherchées, supprimées, modifiées.
	
	Par exemple, imaginons que l'école 
	organise un événement libre et gratuit pour les étudiants.
	Mais pour y assister, ils doivent s'inscrire.
	On veut fournir ce qu'il faut pour :
	\begin{itemize}
	\item inscrire un étudiant ;
	\item vérifier si un étudiant est inscrit ;
	\item désinscrire un étudiant (s'il change d'avis par exemple) ;
	\item afficher la liste des inscrits.
	\end{itemize}

	Pour ce faire,
	on pourra stocker les numéros des étudiants inscrits
	dans un tableau%
	\footnote{%
		Une vraie solution de ce problème
		utiliserait probablement une base de données
		mais ça c'est une autre histoire et un autre cours ;)
	}.
	Comme le tableau ne sera généralement pas rempli,
	on va gérer sa taille logique.
	On a donc deux variables :
	\begin{itemize}
	\item \lda{inscrits} : le tableau des numéros d'étudiants
	\item \lda{nbInscrits} : le nombre d'étudiants actuellement inscrits (la taille logique du tableau)
	\end{itemize}
	
	On va envisager deux cas :
	\begin{enumerate}
	\item
		Les numéros d'étudiants seront placés dans le tableau sans ordre imposé.
	\item
		Les numéros d'étudiants seront placés dans l'ordre.		
	\end{enumerate}
	
	% ==============================================
	\section{Données non triées} 
	% ==============================================
		
		Dans cette approche, 
		les valeurs sont stockées dans le tableau 
		dans un ordre quelconque.
		Par exemple, on pourrait avoir la situation suivante :
		\begin{center}
			\lda{inscrits} = 
			\smallskip
			\begin{tabular}{*{10}{|>{\centering\arraybackslash}m{1cm}}|}
				\hline
				42010 & 41390 & 43342 & 42424 & ? & ? & ? & ? & \dots \\
				\hline
			\end{tabular}
			\smallskip
			\lda{nbInscrits = 4}
		\end{center}
		indiquant qu'il y a pour les moment quatre étudiants inscrits
		dont les numéros sont ceux repris 
		dans les quatre premières cases du tableau.
		
		\subsection{Inscription}
		%---------------------------------------------
		
			Commençons par l'inscription.
			Pour inscrire un étudiant, il suffit de l'ajouter à la suite
			dans le tableau.
			Par exemple, en partant de la situation décrite ci-dessus,
			l'inscription de l'étudiant numéro 42123 aboutit à la situation
			suivante :
			\begin{center}
				\lda{inscrits} = 
				\smallskip
				\begin{tabular}{*{10}{|>{\centering\arraybackslash}m{1cm}}|}
					\hline
					42010 & 41390 & 43342 & 42424 & 42123 & ? & ? & ? & \dots \\
					\hline
				\end{tabular}
				\smallskip
				\lda{nbInscrits = 5}
			\end{center}
			
			L'algorithme est assez simple :
			
			\begin{LDA}
				\LComment{Inscrit un étudiant de numéro donné.}
				\Algo{inscrire}{\Par{inscrits\In\Out}{\Array{n}{entiers}}, \Par{nbInscrits\In\Out}{entier}, \Par{numéro}{entier}}{}
					\LComment{On peut imaginer vérifier que l'étudiant n'est pas déjà inscrit}
					\LComment{On peut imaginer vérifier qu'il reste de la place (c-à-d que le tableau n'est pas plein)}
					\Let inscrits[nbInscrits] \Gets numéro
					\Let nbInscrits \Gets nbInscrits + 1
				\EndAlgo
			\end{LDA}
		
		\subsection{Vérifier une inscription}
		%---------------------------------------------
	
			Pour vérifier si un étudiant est déjà inscrit,
			il faut le rechercher dans la partie utilisée du tableau.
			Cette recherche se fait via un parcours avec sortie
			prématurée. On pourrait se contenter de retourner un
			booléen indiquant si oui ou non on l'a trouvé
			mais retournons plutôt un entier indiquant l'indice où
			il est (-1 si on ne l'a pas trouvé), ce sera plus utile.
			
			\begin{LDA}
				\LComment{Vérifie si un étudiant est inscrit et donne sa position (-1 si non inscrit)}
				\Algo{vérifier}{\Par{inscrits\In}{\Array{n}{entiers}}, \Par{nbInscrits\In}{entier}, \Par{numéro}{entier}}{entier}
					\Decl{i}{entier}
					\Let i \Gets 0
					\While{i < nbInscrits ET inscrits[i] $\ne$ numéro}
						\Let i \Gets i + 1
					\EndWhile
					\If{i < nbInscrits}
						\Return i
					\Else
						\Return -1
					\EndIf
				\EndAlgo
			\end{LDA}
			
			La fiche (A ECRIRE) reprend cet algorithme
			dans un cadre plus général et met en avant
			les variantes et les pièges à éviter.

		\subsection{Désinscription}
		%---------------------------------------------
		
			Pour désinscrire un étudiant,
			il faut le trouver dans le tableau et l'enlever.
			Attention, il ne peut pas y avoir de trou dans un tableau
			(il n'y a pas de case vide).
			Le plus simple est d'y déplacer le dernier élément.
			Par exemple,
			reprenons la situation après inscription de l'étudiant
			numéro 42123.
			La désinscription de l'étudiant numéro 41390 donnerait%
			\footnote{
				Nous avons volontairement indiqué le 42123 en 5\ieme{} position.
				Il est toujours là mais ne sera pas considéré par les
				algorithmes puisque cette case est au-delà de la taille logique
				(donnée par la variable \lda{nbInscrits}).
			}

			\begin{center}
				\lda{inscrits} = 
				\smallskip
				\begin{tabular}{*{10}{|>{\centering\arraybackslash}m{1cm}}|}
					\hline
					42010 & 42123 & 43342 & 42424 & 42123 & ? & ? & ? & \dots \\
					\hline
				\end{tabular}
				\smallskip
				\lda{nbInscrits = 4}
			\end{center}
			
			L'algorithme est assez simple à écrire
			si on utilise la recherche écrite juste avant :
			
			\begin{LDA}
				\LComment{désinscrit l'étudiant donné}
				\Algo{désinscrire}{
						\Par{inscrits\In\Out}{\Array{n}{entiers}}, 
						\Par{nbInscrits\In\Out}{entier}, 
						\\\hfill
						\Par{numéro\In}{entier}
						}{entier}
					\Decl{pos}{entier}
					\Let pos \Gets vérifier(inscrits, nbInscrits, numéro)
					\LComment{On pourrait vérifier et générer une erreur si l'étudiant n'est pas inscrit}
					\Let inscrits[pos] \Gets inscrits[nbInscrits-1]
					\Let nbInscrits \Gets nbInscrits - 1					
				\EndAlgo
			\end{LDA}
			
		\subsection{Exercices}
		%---------------------------------------------
			
			\begin{Exercice}{Éviter la double inscription}
				Comment modifier l'algorithme d'inscription
				pour s'assurer qu'un étudiant ne s'inscrive pas deux fois ?
			\end{Exercice}

			\begin{Exercice}{Limite au nombre d'inscriptions}
				Comment modifier l'algorithme d'inscription
				pour refuser une inscription si le nombre maximal
				de participant est atteint
				en supposant que ce maximum est égal à la taille physique du tableau ?
			\end{Exercice}

			\begin{Exercice}{Vérifier la désinscription}
				Que se passerait-il avec l'algorithme
				de désinscription tel qu'il est
				si on demande à désinscrire un étudiant non inscrit ?
				Que suggérez-vous comme changement ?
			\end{Exercice}

			\begin{Exercice}{Optimiser la désinscription}
				Dans l'algorithme de désinscription tel qu'il est,
				voyez-vous un cas où on effectue une opération inutile ?
				Avez-vous une proposition pour l'éviter ?
			\end{Exercice}

	% ==============================================
	\section{Données triées} 
	% ==============================================

		Imaginons à présent que nous maintenons un ordre
		dans les données du tableau.
		
		Si on reprend l'exemple utilisé pour les données non triées,
		on a :
		\begin{center}
			\lda{inscrits} = 
			\smallskip
			\begin{tabular}{*{10}{|>{\centering\arraybackslash}m{1cm}}|}
				\hline
				41390 & 42010 & 42424 & 43342 & ? & ? & ? & ? & \dots \\
				\hline
			\end{tabular}
			\smallskip
			\lda{nbInscrits = 4}
		\end{center}
		
		Qu'est-ce que ça change au niveau des algorithmes ?
		Beaucoup ! 
		Par exemple,
		plus question de placer un nouvel inscrit à la suite des autres;
		il faut trouver sa place.
		
		\subsection{Rechercher la position d'une inscription}
		%----------------------------------------------------

			Tous les algorithmes que nous allons voir
			dans le cadre de données triées ont une partie en commun :
			la recherche de la position du numéro, s'il est présent
			ou de la position où il aurait dû être s'il est absent.
			Commençons par ça.
			
			L'algorithme est assez proche de celui de la vérification
			d'une inscription vu précédemment si ce n'est
			qu'on peut s'arrêter dès qu'on trouve un numéro plus grand.
			On va aussi ajouter un booléen 
			indiquant si la valeur a été trouvée.

			\begin{LDA}
				\LComment{Recherche un étudiant.}
				\LComment{- trouvé : indique si oui ou non il a été trouvé}
				\LComment{- pos : indique la position où a été trouvé l'étudiant ou la position où il aurait dû être}
				\Algo{rechercher}{
						\Par{inscrits\In}{\Array{n}{entiers}}, 
						\Par{nbInscrits\In}{entier}, 
						\\\hfill\Par{numéro\In}{entier},
						\Par{trouvé\Out}{booléen},
						\Par{pos\Out}{entier}
						}{}
					\Let pos \Gets 0
					\While{pos < nbInscrits ET inscrits[pos] < numéro}
						\Let pos \Gets pos + 1
					\EndWhile
					\Let trouvé \Gets pos < nbInscrits ET inscrits[pos] = numéro
				\EndAlgo
			\end{LDA}
			
			Comprenez-vous pourquoi :
			\begin{itemize}
			\item
				on trouve un < dans la condition du tant que et pas un $\neq$~?
			\item
				l'expression assignée à \lda{trouvé} est composée de deux parties
				et utilise un =~?
			\end{itemize}

		\subsection{Inscrire un étudiant}
		%---------------------------------------------
				
			L'inscription d'un étudiant se fait en trois étapes :
			\begin{enumerate}
			\item
				On recherche l'étudiant via l'algorithme qu'on vient de voir,
				ce qui nous donne la place où le placer.
			\item
				On libère la place pour l'y mettre.
				Comprenez-bien que cette opération n'est pas triviale.
				Si vous tenez des cartes en main, 
				il est tout aussi facile d'y insérer une nouvelle carte
				à n'importe quelle position.
				Avec un tableau,
				il en va tout autrement ;
				pour insérer un élément à un endroit donné,
				il faut décaler tous les suivants d'une position sur la droite.
			\item
				On peut placer l'élément à la position libérée.
			\end{enumerate} 
			
			Ce qui nous donne :
			
			\begin{LDA}
				\LComment{Inscrit un étudiant de numéro donné.}
				\Algo{inscrire}{
						\Par{inscrits\In\Out}{\Array{n}{entiers}}, 
						\Par{nbInscrits\In\Out}{entier}, 
						\Par{numéro\In}{entier}
						}{}
					\Decl{pos}{entier}
					\Decl{trouvé}{booléen}
					\Stmt rechercher( inscrits, nbInscrits, numéro, trouvé, pos )
					\Stmt décalerDroite( inscrits, pos, nbInscrits)
					\Let inscrits[pos] \Gets numéro
					\Let nbInscrits \Gets nbInscrits + 1
				\EndAlgo
				\Empty
				\LComment{Décale d'une position à droite les éléments entre la position début et fin}
				\Algo{décalerDroite}{
						\Par{tab\In\Out}{\Array{n}{entiers}}, 
						\Par{début\In}{entier}, 
						\Par{fin\In}{entier}
						}{}
					\For[-1]{i}{fin}{début}
						\Let tab[i+1] \Gets tab[i]
					\EndFor
				\EndAlgo
			\end{LDA}
		
		\subsection{Vérifier une inscription}
		%---------------------------------------------

			Cette opération est triviale.
			
			\begin{LDA}
				\LComment{Vérifie si un étudiant est inscrit et donne sa position (-1 si non inscrit)}
				\Algo{vérifier}{
						\Par{inscrits\In}{\Array{n}{entiers}}, 
						\Par{nbInscrits\In}{entier}, 
						\\\hfill
						\Par{numéro\In}{entier}
						}{entier}
					\Decl{pos}{entier}
					\Decl{trouvé}{booléen}
					\Stmt rechercher( inscrits, nbInscrits, numéro, trouvé, pos )
					\If{trouvé}
						\Return pos
					\Else
						\Return -1
					\EndIf
				\EndAlgo
			\end{LDA}
			
		\subsection{Désinscription}
		%---------------------------------------------
		
			Ici, il va falloir décaler vers la gauche
			les éléments qui suivent celui à supprimer.

			\begin{LDA}
				\LComment{désinscrit l'étudiant donné}
				\Algo{désinscrire}{
						\Par{inscrits\In\Out}{\Array{n}{entiers}}, 
						\Par{nbInscrits\In\Out}{entier}, 
						\\\hfill
						\Par{numéro\In}{entier}
						}{entier}
					\Decl{pos}{entier}
					\Decl{trouvé}{booléen}
					\Stmt rechercher( inscrits, nbInscrits, numéro, trouvé, pos )
					\Stmt décalerDroite( inscrits, pos+1, nbInscrits)
					\Let nbInscrits \Gets nbInscrits - 1					
				\EndAlgo
				\Empty
				\LComment{Décale d'une position à gauche les éléments entre la position début et fin}
				\Algo{décalerGauche}{
						\Par{tab\In\Out}{\Array{n}{entiers}}, 
						\Par{début\In}{entier}, 
						\Par{fin\In}{entier}
						}{}
					\For{i}{début}{fin}
						\Let tab[i-1] \Gets tab[i]
					\EndFor
				\EndAlgo
			\end{LDA}
			
		\subsection{Exercices}
		%---------------------------------------------
			
			\begin{Exercice}{Éviter la double inscription}
				Comment modifier l'algorithme d'inscription
				pour s'assurer qu'un étudiant ne s'inscrive pas deux fois ?
			\end{Exercice}

			\begin{Exercice}{Limite au nombre d'inscriptions}
				Comment modifier l'algorithme d'inscription
				pour refuser une inscription si le nombre maximal
				de participant est atteint
				en supposant que ce maximum est égal à la taille physique du tableau ?
			\end{Exercice}

			\begin{Exercice}{Vérifier la déinscription}
				Que se passerait-il avec l'algorithme
				de désinscription tel qu'il est
				si on demande à désinscrire un étudiant non inscrit ?
				Que suggérez-vous comme changement ?
			\end{Exercice}

		\subsection{Intérêt de trier les valeurs ?}
		%---------------------------------------------

			Est-ce que trier les valeurs est vraiment intéressant ?
			On voit que la recherche est un peu plus rapide
			(en moyenne 2x plus rapide si l'élément ne s'y trouve pas).
			Par contre, l'ajout et la suppression d'un élément
			sont plus lents.
			En plus, les algorithmes sont plus complexes à écrire
			et à comprendre.
			Le gain ne parait pas évident
			et en effet, en l'état, il ne l'est pas.
			
			Mais c'est sans compter 
			un autre algorithme de recherche, beaucoup plus rapide,
			la recherche dichotomique, que nous allons voir maintenant.
			
	% ==============================================
	\section{La recherche dichotomique\index{recherche dichotomique}} 
	% ==============================================

		La recherche dichotomique 
		est un algorithme de recherche d'une valeur dans un tableau trié.
		Il a pour essence de réduire à chaque étape 
		la taille de l’ensemble de recherche de moitié, 
		jusqu’à ce qu’il ne reste qu’un seul élément 
		dont la valeur devrait être celle recherchée, 
		sauf bien entendu en cas d’inexistence de cette valeur 
		dans l’ensemble de départ. 
	
		Pour l'expliquer,
		on va prendre un tableau d'entiers complètement rempli.
		Il sera facile d'adapter la solution à un tableau partiellement
		rempli (avec une taille logique) 
		ou un tableau contenant tout autre type de valeurs
		qui peut s'ordonner.
		
		{\sffamily\bfseries\upshape
		Description de l’algorithme}
	
		Soit \lda{val} la valeur recherchée dans une zone délimitée 
		par les indices \lda{indiceGauche} et \lda{indiceDroit}. 
		On commence par déterminer l’élément médian, 
		c’est-à-dire celui qui se trouve « au milieu » 
		de la zone de recherche%
		\footnote{%
			Cet élément médian n'est pas tout à fait au milieu 
			dans le cas d'une zone contenant un nombre pair d'éléments.
		} ; 
		son indice sera déterminé par la formule
	
		{\centering
		\lda{indiceMédian \Gets (indiceGauche + indiceDroit) DIV 2}\par{}
		}
	
		On compare alors \lda{val} avec la valeur de cet élément médian ; 
		il est possible qu’on ait trouvé la valeur cherchée; 
		sinon, on partage la zone de recherche en deux parties~: 
		une qui \textbf{ne contient certainement pas} la valeur cherchée 
		et une qui \textbf{pourrait la contenir}. 
		C’est cette deuxième partie qui devient la nouvelle zone de recherche. 
		On adapte \lda{indiceGauche} ou \lda{indiceDroit} en conséquence.
		On réitère le processus jusqu’à ce que la valeur cherchée soit trouvée ou
		que la zone de recherche soit réduite à sa plus simple expression,
		c’est-à-dire un seul élément.
			
		{\sffamily\bfseries\upshape
		Exemple de recherche fructueuse}

		Supposons que l’on recherche la valeur \textbf{23} 
		dans un tableau de 20 entiers. 
		La zone ombrée représente à chaque fois la partie de recherche, 
		qui est au départ le tableau dans son entièreté.
		Au départ,
		\lda{indiceGauche} vaut 0 et
		\lda{indiceDroit} vaut 19.

		\begin{center}
		\scriptsize
		\begin{tabular}{*{20}{>{\centering\sffamily\itshape\arraybackslash}m{2.5mm}}}
		 \textcolor{gray}{0} &
		 \textcolor{gray}{1} &
		 \textcolor{gray}{2} &
		 \textcolor{gray}{3} &
		 \textcolor{gray}{4} &
		 \textcolor{gray}{5} &
		 \textcolor{gray}{6} &
		 \textcolor{gray}{7} &
		 \textcolor{gray}{8} &
		 \textcolor{gray}{9} &
		 \textcolor{gray}{10} &
		 \textcolor{gray}{11} &
		 \textcolor{gray}{12} &
		 \textcolor{gray}{13} &
		 \textcolor{gray}{14} &
		 \textcolor{gray}{15} &
		 \textcolor{gray}{16} &
		 \textcolor{gray}{17} &
		 \textcolor{gray}{18} &
		 \textcolor{gray}{19}
			 \\
		\end{tabular}
		\begin{tabular}{|*{20}{>{\centering\arraybackslash}m{2.5mm}|}}
			\hline
			{\cellcolor{gray!25}  1} &
			{\cellcolor{gray!25}  3} &
			{\cellcolor{gray!25}  5} &
			{\cellcolor{gray!25}  7} &
			{\cellcolor{gray!25}  7} &
			{\cellcolor{gray!25}  9} &
			{\cellcolor{gray!25}  9} &
			{\cellcolor{gray!25} 10} &
			{\cellcolor{gray!25} 10} &
			{\cellcolor{gray!25} 15} &
			{\cellcolor{gray!25} 16} &
			{\cellcolor{gray!25} 20} &
			{\cellcolor{gray!25} 23} &
			{\cellcolor{gray!25} 23} &
			{\cellcolor{gray!25} 25} &
			{\cellcolor{gray!25} 28} &
			{\cellcolor{gray!25} 28} &
			{\cellcolor{gray!25} 28} &
			{\cellcolor{gray!25} 29} &
			{\cellcolor{gray!25} 29}\\\hline
		\end{tabular}
		\end{center}
		
		\bigskip

		\textit{Première étape}~:
		\lda{indiceMédian \Gets (0 + 19) DIV 2}, c’est-à-dire 9. 
		Comme la valeur en position 9 est 15, 
		la valeur cherchée doit se trouver à sa droite, et
		on prend comme nouvelle zone de recherche celle délimitée par
		\lda{indiceGauche} qui vaut 10 et \lda{indiceDroit} qui vaut 19.
		
		\bigskip
		
		\begin{center}
		\scriptsize
		\begin{tabular}{*{20}{>{\centering\sffamily\itshape\arraybackslash}m{2.5mm}}}
		 \textcolor{gray}{0} &
		 \textcolor{gray}{1} &
		 \textcolor{gray}{2} &
		 \textcolor{gray}{3} &
		 \textcolor{gray}{4} &
		 \textcolor{gray}{5} &
		 \textcolor{gray}{6} &
		 \textcolor{gray}{7} &
		 \textcolor{gray}{8} &
		 \textcolor{gray}{9} &
		 \textcolor{gray}{10} &
		 \textcolor{gray}{11} &
		 \textcolor{gray}{12} &
		 \textcolor{gray}{13} &
		 \textcolor{gray}{14} &
		 \textcolor{gray}{15} &
		 \textcolor{gray}{16} &
		 \textcolor{gray}{17} &
		 \textcolor{gray}{18} &
		 \textcolor{gray}{19}
			 \\
		\end{tabular}
		\begin{tabular}{|*{20}{>{\centering\arraybackslash}m{2.5mm}|}}
			\hline
			{ 1} &
			{  3} &
			{  5} &
			{  7} &
			{  7} &
			{  9} &
			{  9} &
			{ 10} &
			{ 10} &
			{ 15} &
			{\cellcolor{gray!25} 16} &
			{\cellcolor{gray!25} 20} &
			{\cellcolor{gray!25} 23} &
			{\cellcolor{gray!25} 23} &
			{\cellcolor{gray!25} 25} &
			{\cellcolor{gray!25} 28} &
			{\cellcolor{gray!25} 28} &
			{\cellcolor{gray!25} 28} &
			{\cellcolor{gray!25} 29} &
			{\cellcolor{gray!25} 29}\\\hline
		\end{tabular}
		\end{center}

		\bigskip

		\textit{Deuxième étape}~:
		\lda{indiceMédian \Gets (10 + 19) DIV 2}, c’est-à-dire 14. 
		Comme on y trouve la valeur 25, 
		on garde les éléments situées à la gauche de celui-ci ; 
		la nouvelle zone de recherche est [10, 13].
				
		\begin{center}
		\scriptsize
		\begin{tabular}{*{20}{>{\centering\sffamily\itshape\arraybackslash}m{2.5mm}}}
		 \textcolor{gray}{0} &
		 \textcolor{gray}{1} &
		 \textcolor{gray}{2} &
		 \textcolor{gray}{3} &
		 \textcolor{gray}{4} &
		 \textcolor{gray}{5} &
		 \textcolor{gray}{6} &
		 \textcolor{gray}{7} &
		 \textcolor{gray}{8} &
		 \textcolor{gray}{9} &
		 \textcolor{gray}{10} &
		 \textcolor{gray}{11} &
		 \textcolor{gray}{12} &
		 \textcolor{gray}{13} &
		 \textcolor{gray}{14} &
		 \textcolor{gray}{15} &
		 \textcolor{gray}{16} &
		 \textcolor{gray}{17} &
		 \textcolor{gray}{18} &
		 \textcolor{gray}{19}
			 \\
		\end{tabular}
		\begin{tabular}{|*{20}{>{\centering\arraybackslash}m{2.5mm}|}}
			\hline
			{ 1} &
			{  3} &
			{  5} &
			{  7} &
			{  7} &
			{  9} &
			{  9} &
			{ 10} &
			{ 10} &
			{ 15} &
			{\cellcolor{gray!25} 16} &
			{\cellcolor{gray!25} 20} &
			{\cellcolor{gray!25} 23} &
			{\cellcolor{gray!25} 23} &
			{ 25} &
			{ 28} &
			{ 28} &
			{ 28} &
			{ 29} &
			{ 29}\\\hline
		\end{tabular}
		\end{center}

		\bigskip

		\textit{Troisième étape}~:
		\lda{indiceMédian \Gets (10 + 13) DIV 2}, c’est-à-dire 11 
		où se trouve l’élément 20. 
		La zone de recherche devient
		\lda{indiceGauche} vaut 12 et \lda{indiceDroit} vaut 13.

		
		\begin{center}
		\scriptsize
		\begin{tabular}{*{20}{>{\centering\sffamily\itshape\arraybackslash}m{2.5mm}}}
		 \textcolor{gray}{0} &
		 \textcolor{gray}{1} &
		 \textcolor{gray}{2} &
		 \textcolor{gray}{3} &
		 \textcolor{gray}{4} &
		 \textcolor{gray}{5} &
		 \textcolor{gray}{6} &
		 \textcolor{gray}{7} &
		 \textcolor{gray}{8} &
		 \textcolor{gray}{9} &
		 \textcolor{gray}{10} &
		 \textcolor{gray}{11} &
		 \textcolor{gray}{12} &
		 \textcolor{gray}{13} &
		 \textcolor{gray}{14} &
		 \textcolor{gray}{15} &
		 \textcolor{gray}{16} &
		 \textcolor{gray}{17} &
		 \textcolor{gray}{18} &
		 \textcolor{gray}{19}
			 \\
		\end{tabular}
		\begin{tabular}{|*{20}{>{\centering\arraybackslash}m{2.5mm}|}}
			\hline
			{ 1} &
			{  3} &
			{  5} &
			{  7} &
			{  7} &
			{  9} &
			{  9} &
			{ 10} &
			{ 10} &
			{ 15} &
			{ 16} &
			{ 20} &
			{\cellcolor{gray!25} 23} &
			{\cellcolor{gray!25} 23} &
			{ 25} &
			{ 28} &
			{ 28} &
			{ 28} &
			{ 29} &
			{ 29}\\\hline
		\end{tabular}
		\end{center}

		\bigskip

		\textit{Quatrième étape~:}
		\lda{indiceMédian \Gets (12 + 13) DIV 2}, 
		c’est-à-dire 12 où se trouve la valeur cherchée ; 
		la recherche est terminée.

	{\sffamily\bfseries\upshape
	Exemple de recherche infructueuse}

		Recherchons à présent la valeur \textbf{8}. 
		Les étapes de la recherche vont donner successivement
		
		\begin{center}
		\scriptsize
		\begin{tabular}{*{20}{>{\centering\sffamily\itshape\arraybackslash}m{2.5mm}}}
		 \textcolor{gray}{0} &
		 \textcolor{gray}{1} &
		 \textcolor{gray}{2} &
		 \textcolor{gray}{3} &
		 \textcolor{gray}{4} &
		 \textcolor{gray}{5} &
		 \textcolor{gray}{6} &
		 \textcolor{gray}{7} &
		 \textcolor{gray}{8} &
		 \textcolor{gray}{9} &
		 \textcolor{gray}{10} &
		 \textcolor{gray}{11} &
		 \textcolor{gray}{12} &
		 \textcolor{gray}{13} &
		 \textcolor{gray}{14} &
		 \textcolor{gray}{15} &
		 \textcolor{gray}{16} &
		 \textcolor{gray}{17} &
		 \textcolor{gray}{18} &
		 \textcolor{gray}{19}
			 \\
		\end{tabular}
		\begin{tabular}{|*{20}{>{\centering\arraybackslash}m{2.5mm}|}}
			\hline
			{\cellcolor{gray!25}  1} &
			{\cellcolor{gray!25}  3} &
			{\cellcolor{gray!25}  5} &
			{\cellcolor{gray!25}  7} &
			{\cellcolor{gray!25}  7} &
			{\cellcolor{gray!25}  9} &
			{\cellcolor{gray!25}  9} &
			{\cellcolor{gray!25} 10} &
			{\cellcolor{gray!25} 10} &
			{\cellcolor{gray!25} 15} &
			{\cellcolor{gray!25} 16} &
			{\cellcolor{gray!25} 20} &
			{\cellcolor{gray!25} 23} &
			{\cellcolor{gray!25} 23} &
			{\cellcolor{gray!25} 25} &
			{\cellcolor{gray!25} 28} &
			{\cellcolor{gray!25} 28} &
			{\cellcolor{gray!25} 28} &
			{\cellcolor{gray!25} 29} &
			{\cellcolor{gray!25} 29}\\\hline
		\end{tabular}
		\end{center}

		\begin{center}
		\scriptsize
		\begin{tabular}{|*{20}{>{\centering\arraybackslash}m{2.5mm}|}}
			\hline
			{\cellcolor{gray!25}  1} &
			{\cellcolor{gray!25}  3} &
			{\cellcolor{gray!25}  5} &
			{\cellcolor{gray!25}  7} &
			{\cellcolor{gray!25}  7} &
			{\cellcolor{gray!25}  9} &
			{\cellcolor{gray!25}  9} &
			{\cellcolor{gray!25} 10} &
			{\cellcolor{gray!25} 10} &
			{ 15} &
			{ 16} &
			{ 20} &
			{ 23} &
			{ 23} &
			{ 25} &
			{ 28} &
			{ 28} &
			{ 28} &
			{ 29} &
			{ 29}\\\hline
		\end{tabular}
		\end{center}

		\begin{center}
		\scriptsize
		\begin{tabular}{|*{20}{>{\centering\arraybackslash}m{2.5mm}|}}
			\hline
			{ 1} &
			{  3} &
			{  5} &
			{  7} &
			{  7} &
			{\cellcolor{gray!25}  9} &
			{\cellcolor{gray!25}  9} &
			{\cellcolor{gray!25} 10} &
			{\cellcolor{gray!25} 10} &
			{ 15} &
			{ 16} &
			{ 20} &
			{ 23} &
			{ 23} &
			{ 25} &
			{ 28} &
			{ 28} &
			{ 28} &
			{ 29} &
			{ 29}\\\hline
		\end{tabular}
		\end{center}

		\begin{center}
		\scriptsize
		\begin{tabular}{|*{20}{>{\centering\arraybackslash}m{2.5mm}|}}
			\hline
			{ 1} &
			{  3} &
			{  5} &
			{  7} &
			{  7} &
			{\cellcolor{gray!25}  9} &
			{  9} &
			{ 10} &
			{ 10} &
			{ 15} &
			{ 16} &
			{ 20} &
			{ 23} &
			{ 23} &
			{ 25} &
			{ 28} &
			{ 28} &
			{ 28} &
			{ 29} &
			{ 29}\\\hline
		\end{tabular}
		\end{center}

		\smallskip

		Arrivé à ce stade, 
		la zone de recherche s’est réduite à un seul élément.
		Ce n’est pas celui qu’on recherche 
		mais c’est à cet endroit qu’il se serait trouvé ; 
		c’est donc là qu’on pourra éventuellement l'insérer. 
		Le paramètre de sortie prend la valeur 5.

		{\sffamily\bfseries
		Algorithme}
		
		\begin{LDA}
			\Algo{rechercheDichotomique}{
					\\\hfill
					\Par{tab\In}{\Array{n}{entiers}}, 
					\Par{valeur\In}{entier}, 
					\Par{pos\Out}{entier}
					}{booléen}
				\Decl{indiceDroit, indiceGauche, indiceMédian}{entiers}
				\Decl{candidat}{T}
				\Decl{trouvé}{booléen}
				\Empty
				\Let indiceGauche \Gets 0
				\Let indiceDroit \Gets n-1
				\Let trouvé \Gets faux
				\Empty
				\While{NON trouvé ET indiceGauche {${\leq}$} indiceDroit}
					\Let indiceMédian \Gets (indiceGauche + indiceDroit) DIV 2
					\Let candidat \Gets tab[indiceMédian]
					\If{candidat = valeur} 
						\Let trouvé \Gets vrai
					\ElsIf{candidat < valeur}
						\Let indiceGauche \Gets indiceMédian + 1
						\RComment on garde la partie droite
					\Else
						\Let indiceDroit \Gets indiceMédian – 1
						\RComment on garde la partie gauche
					\EndIf
				\EndWhile
				\Empty
				\If{trouvé}
					\Let pos \Gets indiceMédian
				\Else
					\Let pos \Gets indiceGauche
					\RComment dans le cas où la valeur n’est pas trouvée,
					\Empty 
					\RComment on vérifiera que indiceGauche donne la valeur où elle pourrait être insérée.
				\EndIf
				\Empty
				\Return trouvé
			\EndAlgo
		\end{LDA}

	% ==============================================
	\section{Introduction à la complexité} 
	% ==============================================

		L’algorithme de recherche dichotomique 
		est beaucoup plus rapide que l’algorithme de recherche linéaire. 
		Mais qu’est-ce que cela veut dire exactement ? 
		En est-on sûr ? 
		Comment le mesure-t-on ? 
		Quels critères utilise-t-on ?
		De façon générale, comment comparer la vitesse
		de deux algorithmes différents qui résolvent le même problème.
		
		\subsection{Une approche pratique : la simulation numérique}
		%--------------------------------------------------
	
			On pourrait se dire que pour comparer deux algorithmes, 
			il suffit de les traduire tous les deux 
			dans un langage de programmation, 
			de les exécuter et de comparer leur temps d’exécution. 
			Cette technique pose toutefois quelques problèmes :			
			\begin{itemize}
				\item 
					Il faut que les programmes soient exécutés 
					dans des environnements strictement identiques 
					ce qui n’est pas toujours le cas ou facile à vérifier.
				\item
					Certains environnements peuvent être favorables
					à un algorithme par rapport à l'autre
					ce qui ne serait pas le cas d'un autre environnement.
					Par exemple, certaines architectures
					sont plus rapides dans les calculs entiers
					mais moins dans les calculs flottants.
				\item 
					Le test ne porte que sur un (voir quelques uns) jeu(x) de tests. 
					Comment en tirer un enseignement général ? 
					Que se passerait-il avec des données plus importantes ? 
					Avec des données différentes ?
			\end{itemize}		
			En fait, un chiffre précis ne nous intéresse pas. 
			Ce qui est vraiment intéressant,
			c'est de savoir comment l'algorithme va se comporter
			avec de grandes données.
			C'est ce qu'apporte l'approche suivante.
		
		\subsection{Une approche théorique : la complexité}
		%--------------------------------------------------
		
			Avec cette approche,
			on va déduire de l'algorithme
			le nombre d’opérations de base à effectuer
			en fonction de la \textbf{taille} du problème à résoudre
			et en déduire 
			comment il va se comporter sur de «~gros~» problèmes ?
		
			Dans le cas de la recherche dans un tableau, 
			la taille du problème 
			est la taille du tableau dans lequel on recherche un élément. 
			On peut considérer que l’opération de base est la comparaison 
			avec un élément du tableau. 
			La question est alors la suivante~: 
			pour un tableau de taille $n$, 
			à combien de comparaisons faut-il procéder 
			pour trouver l’élément (ou se rendre compte de son absence) ?
		
			{\bfseries
			Pour la recherche linéaire}
		
				Cela dépend évidemment de la position de la valeur à trouver. 
				Dans le meilleur des cas c’est 1, 
				dans le pire c’est $n$ mais on peut dire, 
				qu’en moyenne, 
				cela entraine «~$n/2$~» comparaisons 
				(que ce soit pour la trouver ou se rendre compte de son absence).
		
			{\bfseries
			Pour la recherche dichotomique}
		
				Ici, la zone de recherche est divisée par 2, à chaque étape. 
				Imaginons une liste de 64 éléments~: 
				après 6 étapes, on arrive à un élément isolé. 
				Pour une liste de taille $n$, 
				on peut en déduire que le nombre de comparaisons 
				est au maximum l'exposant qu'il faut donner à 2 pour obtenir $n$, 
				soit «~$\log_2(n)$~».
		
			{\sffamily\bfseries\upshape
			Comparaisons}
		
			Voici un tableau comparatif du nombre de comparaison 
			en fonction de la taille $n$
		
			\begin{center}
			\begin{tabular}{
					|>{\raggedright\arraybackslash}m{4.4300003cm}
					|>{\raggedleft\arraybackslash}m{0.507cm}
					|>{\raggedleft\arraybackslash}m{0.71900004cm}
					|>{\raggedleft\arraybackslash}m{0.93100005cm}
					|>{\raggedleft\arraybackslash}m{1.2479999cm}
					|>{\raggedleft\arraybackslash}m{1.4599999cm}
					|>{\raggedleft\arraybackslash}m{1.671cm}|}
			\hline
			$n$ & 10 & 100 & 1000 & 10.000 & 100.000 & 1 million \\
			\hline
			recherche linéaire & 5 & 50 & 500 & 5.000 & 50.000 & 500.000 \\
			\hline
			recherche dichotomique & 4 & 7 & 10 & 14 & 17 & 20 \\
			\hline
			\end{tabular}
			\end{center}
			
			On voit que c’est surtout pour des grands problèmes 
			que la recherche dichotomique montre toute son efficacité. 
			On voit aussi que ce qui est important pour mesurer la complexité 
			c'est l'ordre de grandeur du nombre de comparaisons. 
		
			On dira que la recherche simple 
			est un algorithme de complexité linéaire
			(c'est-à-dire que le nombre d'opérations est de l'ordre de $n$ 
			ou proportionnel à $n$ ou encore que si on double la taille
			du problème, le temps va aussi être doublé) 
			ce qu’on note en langage plus mathématique $O(n)$
			(prononcé «~grand $O$ de $n$~»). 
			
			Pour la recherche dichotomique, la complexité est logarithmique, 
			et on le note $O(\log_2(n))$ ce qui veut dire
			que si on double la taille du problème, 
			on a juste une itération en plus dans la boucle.
			
			Comparons les complexités les plus courantes.
			\begin{center}
			\begin{tabular}{
				|>{\centering\arraybackslash}m{2cm}
				|>{\raggedright\arraybackslash}m{0.8cm}
				|>{\raggedright\arraybackslash}m{0.8cm}
				|>{\raggedright\arraybackslash}m{1.0cm}
				|>{\raggedright\arraybackslash}m{1.3cm}
				|>{\raggedright\arraybackslash}m{1.287cm}
				|>{\raggedright\arraybackslash}m{1.425cm}
				|>{\raggedright\arraybackslash}m{1.714cm}|}
			\hline
			$n$ & 10 & 100 & 1000 & $10^4$ & $10^5$ & $10^6$ & $10^9$ \\
			\hline
			$O(1)$ & 1 & 1 & 1 & 1 & 1 & 1 & 1\\
			\hline
			$O(\log_2(n))$ & 4 & 7 & 10 & 14 & 17 & 20 & 30\\
			\hline
			$O(n)$ & 10 & 100 & 1000 & $10^4$ & $10^5$ & $10^6$ & $10^9$ \\
			\hline
			$O(n^2)$ & 100 & $10^4$ & $10^6$ & $10^8$ & $10^10$ & $10^{12}$ & $10^{18}$ \\
			\hline
			$O(n^3)$ & 1000 & $10^6$ & $10^9$ & $10^{12}$ & $10^{15}$ & $10^{18}$ & $10^{27}$ \\
			\hline
			$O(2^n)$ & 1024 & $10^{30}$ & $10^{301}$ & $10^{3010}$ & $10^{30102}$ 
					 & $10^{301029}$ & $10^{301029995}$ \\
			\hline
			\end{tabular}
			\end{center}
			
			On voit ainsi que si on trouve un algorithme correct 
			pour résoudre un problème mais que cet algorithme 
			est de complexité exponentielle 
			alors cet algorithme ne sert à rien en pratique. 
			Par exemple un algorithme de recherche de complexité exponentielle, 
			exécuté sur une machine pouvant effectuer un milliard de comparaisons par secondes, 
			prendrait plus de dix mille milliards d’années 
			pour trouver une valeur dans un tableau de 100 éléments.

			\begin{Exercice}{Calcul de complexités}
				Quelle est la complexité d'un algorithme qui :		
				\begin{enumerate}[label=\alph*)]
				\item 
					recherche le maximum d'un tableau de $n$ éléments ?
				\item 
					remplace par 0 toutes les occurrences du maximum 
					d'un tableau de $n$ éléments ?
				\item 
					vérifie si un tableau contient deux éléments égaux ?
				\item 
					trie par recherche des minima successifs ?
				\item 
					vérifie si les éléments d'un tableau forment un palindrome ?
				\item
					cherche un élément dans un tableau 
					en essayant des cases au hasard jusqu'à le trouver ?
				\end{enumerate}
			\end{Exercice}

			\begin{Exercice}{Gestion des données}
				Comparez les algorithmes d'ajout, 
				de recherche et de suppression
				de valeurs dans le cas d'un tableau trié
				et d'un tableau non trié.
			\end{Exercice}
			
			\begin{Exercice}{Réflexion}
				L’algorithme de recherche dichotomique 
				est-il toujours à préférer à
				l’algorithme de recherche linéaire ?
			\end{Exercice}
	
