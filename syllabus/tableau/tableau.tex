%======================
\chapter{Les tableaux}
%======================

	\marginicon{objectif}
	Dans ce chapitre nous étudions les tableaux, 
	une structure qui peut contenir 
	plusieurs exemplaires de données similaires.

	% =============================
	\section{Utilité des tableaux}
	% =============================
	
		Nous allons introduire la notion de tableau à partir d’un exemple 
		dans lequel l’utilité de cette structure de données 
		apparaitra de façon naturelle.
	
		\textbf{Exemple.}
		Statistiques de ventes.
		\begin{quote}	
			Un gérant d’une entreprise commerciale 
			souhaite connaitre l’impact 
			d’une journée de promotion publicitaire 
			sur la vente de dix de ses produits.
			Pour ce faire, les numéros de ces produits 
			(numérotés de 0 à 9 pour simplifier) 
			ainsi que les quantités vendues 
			pendant cette journée de promotion 
			sont encodés au fur et à mesure de leurs ventes. 
			En fin de journée, 
			le vendeur entrera la valeur -1 
			pour signaler la fin de l’introduction des données. 
			Ensuite, les statistiques des ventes seront affichées.
		\end{quote}
	
		La démarche générale se décompose en trois parties~:
		\begin{itemize}
		\item
			le traitement de début de journée, qui consiste essentiellement à mettre
			les compteurs des quantités vendues pour chaque produit à 0~;
		\item
			le traitement itératif durant toute la journée~: 
			au fur et à mesure des ventes, 
			il convient de les enregistrer, 
			c’est-à-dire d’ajouter au compteur des ventes d’un produit 
			la quantité vendue de ce produit~; 
			ce traitement itératif s’interrompra 
			lorsque la valeur -1 sera introduite~;
		\item
			le traitement final, consistant à communiquer les valeurs des compteurs
			pour chaque produit.
		\end{itemize}
	
		Vous trouverez sur la page suivante
		une version possible de cet algorithme.
	
		\begin{LDA}
		\LComment Calcule et affiche la quantité vendue de 10 produits.
		\Algo{statistiquesVentesSansTableau}{}{}
			\Empty
			\Decl{cpt0, cpt1, cpt2, cpt3, cpt4, cpt5, cpt6, cpt7, cpt8, cpt9}{entiers}
			\Decl{numéroProduit, quantité}{entiers}
			\Empty
			\Let cpt0 \Gets 0
			\Let cpt1 \Gets 0
			\Let cpt2 \Gets 0
			\Let cpt3 \Gets 0
			\Let cpt4 \Gets 0
			\Let cpt5 \Gets 0
			\Let cpt6 \Gets 0
			\Let cpt7 \Gets 0
			\Let cpt8 \Gets 0
			\Let cpt9 \Gets 0
			\Empty
			\Write "Introduisez le numéro du produit~:"
			\Read numéroProduit
			\Empty
			\While{numéroProduit $\neq$ -1}
			\Empty
				\Write "Introduisez la quantité vendue~:"
				\Read quantité
				\Empty
				\Switch{numéroProduit \K{vaut}}
					\Stmt 0~: cpt0 \Gets cpt0 + quantité
					\Stmt 1~: cpt1 \Gets cpt1 + quantité
					\Stmt 2~: cpt2 \Gets cpt2 + quantité
					\Stmt 3~: cpt3 \Gets cpt3 + quantité
					\Stmt 4~: cpt4 \Gets cpt4 + quantité
					\Stmt 5~: cpt5 \Gets cpt5 + quantité
					\Stmt 6~: cpt6 \Gets cpt6 + quantité
					\Stmt 7~: cpt7 \Gets cpt7 + quantité
					\Stmt 8~: cpt8 \Gets cpt8 + quantité
					\Stmt 9~: cpt9 \Gets cpt9 + quantité
				\EndSwitch
				\Empty
				\Write "Introduisez le numéro du produit~:"
				\Read numéroProduit
				\Empty
			\EndWhile
			\Empty
			\Write "quantité vendue de produit 0~:", cpt0
			\Write "quantité vendue de produit 1~:", cpt1
			\Write "quantité vendue de produit 2~:", cpt2
			\Write "quantité vendue de produit 3~:", cpt3
			\Write "quantité vendue de produit 4~:", cpt4
			\Write "quantité vendue de produit 5~:", cpt5
			\Write "quantité vendue de produit 6~:", cpt6
			\Write "quantité vendue de produit 7~:", cpt7
			\Write "quantité vendue de produit 8~:", cpt8
			\Write "quantité vendue de produit 9~:", cpt9
			\Empty
		\EndAlgo
		\end{LDA}
	
		Il est clair, à la lecture de cet algorithme, 
		qu’une simplification d’écriture s’impose~! 
		Et que ce passerait-il si le nombre de produits à
		traiter était de 20 ou 100~? 
		Le but de l’informatique étant de dispenser l’humain 
		des tâches répétitives, 
		le programmeur peut en espérer autant 
		de la part d’un langage de programmation~!
		La solution est apportée par un nouveau type de variables~: 
		les \textbf{variables indicées} ou \textbf{tableaux}.
	
		Au lieu d’avoir à manier dix compteurs distincts
		(\lda{cpt0}, \lda{cpt1}, etc.), 
		nous allons envisager une seule «~grande~» variable 
		\lda{cpt} compartimentée en dix «~cases~» ou «~sous-variables~»
		(qu’on appelle aussi les «~éléments~» du tableau). 
		Elles se distingueront les unes des autres par un numéro 
		(on dit un «~indice~»)~: 
		\lda{cpt0} deviendrait ainsi \lda{cpt[0]}, 
		\lda{cpt1} deviendrait \lda{cpt[1]}, 
		et ainsi de suite jusqu’à
		\lda{cpt9} qui deviendrait \lda{cpt[9]}.
	
		\begin{center}
			\begin{tabular}{*{11}{>{\centering\arraybackslash}m{7mm}}}
				{} &
				\lda{cpt[0]} &
				\lda{cpt[1]} &
				\lda{cpt[2]} &
				\lda{cpt[3]} &
				\lda{cpt[4]} &
				\lda{cpt[5]} &
				\lda{cpt[6]} &
				\lda{cpt[7]} &
				\lda{cpt[8]} &
				\lda{cpt[9]} 
				\\\hhline{~*{10}{-}}
				\multicolumn{1}{m{7mm}|}{\lda{cpt}} &
				\multicolumn{1}{m{7mm}|}{~} &
				\multicolumn{1}{m{7mm}|}{~} &
				\multicolumn{1}{m{7mm}|}{~} &
				\multicolumn{1}{m{7mm}|}{~} &
				\multicolumn{1}{m{7mm}|}{~} &
				\multicolumn{1}{m{7mm}|}{~} &
				\multicolumn{1}{m{7mm}|}{~} &
				\multicolumn{1}{m{7mm}|}{~} &
				\multicolumn{1}{m{7mm}|}{~} &
				\multicolumn{1}{m{7mm}|}{~}
				\\\hhline{~*{10}{-}}
			\end{tabular}
		\end{center}
	
		Un des intérêts de cette notation 
		est la possibilité de faire apparaitre
		une variable entre les crochets, 
		par exemple \lda{cpt[i]}, 
		ce qui permet une grande économie de lignes de code.
		
		Voici la version avec tableau.
	
		\label{tableau:tab1DStock10Articles}
		\begin{LDA}
		\label{tableau:tab1DStock10Articles}
		\LComment Calcule et affiche la quantité vendue de 10 produits.
		\Algo{statistiquesVentesAvecTableau}{}{}
			\Empty
			\Decl{cpt}{\Array{10}{entiers}}
			\Decl{i, numéroProduit, quantité}{entiers}
			\Empty
			\For{i}{0}{9}
				\Let cpt[i] \Gets 0
			\EndFor
			\Empty
			\Write "Introduisez le numéro du produit~:"
			\Read numéroProduit
			\Empty
			\While{numéroProduit $\neq$ -1}
			\Empty
				\Write "Introduisez la quantité vendue~:"
				\Read quantité
				\Empty
				\Let cpt[numéroProduit] \Gets cpt[numéroProduit] + quantité
				\Empty
				\Write "Introduisez le numéro du produit~:"
				\Read numéroProduit
				\Empty
			\EndWhile
			\Empty
			\For{i}{0}{9}
				\Write "quantité vendue de produit ", i, ": ", cpt[i]
			\EndFor
			\Empty
		\EndAlgo
		\end{LDA}
		
	% ====================
	\section{Définitions}
	% ====================
	
		\marginicon{definition}\index{tableau}
		Un \textbf{tableau} est une suite d’éléments de même type 
		portant tous le même nom mais se distinguant 
		les uns des autres par un indice.
	
		L’\textbf{indice} est un entier\index{tableau (indice)}
		donnant la position d’un élément dans la suite. 
		Cet indice varie entre la position du premier élément 
		et la position du dernier élément, 
		ces positions correspondant aux bornes de l’indice.
		Notons qu’il n’y a pas de «~trou~»~: 
		tous les éléments existent entre le premier et le dernier indice.
	
		La \textbf{taille} d’un tableau\index{tableau (taille)}
		est le nombre de ses éléments.
		Attention~! la taille d’un tableau ne peut pas être modifiée pendant
		son utilisation.
	
	% ==================
	\section{Déclaration}
	% ==================
	
		\marginicon{definition}
		Pour déclarer un tableau, on écrit~:
	
		\begin{LDA}
			\Decl{nomTableau}{\Array{\textit{taille}}{\textit{TypeElément}}}
		\end{LDA}

		où \lda{taille} est le nombre d’éléments 
		et \lda{TypeElément} est le type des éléments 
			que l’on trouvera dans le tableau.
			Tous les types sont permis
			mais tous les éléments
			sont du même type. 
		
	% ==================
	\section{Utilisation}
	% ==================
			
		Un élément (une case) du tableau peut être accédé
		via la notation
		\begin{LDA}
			\Stmt tab[i]	\Comment{élément du tableau 'tab' à l’indice i}
		\end{LDA}
		La première case du tableau porte l’indice \lda{0}%
		\footnote{%
			Nous considérons que la première case du tableau
			porte le numéro 0 comme c’est le cas dans beaucoup de langages
			de programmation (comme Java par exemple).
			Plus loin, nous verrons une notation alternative
			qui permet de choisir un autre numéro de début pour le tableau,
			plus naturel pour certains problèmes.
		},
		la dernière l’indice \lda{taille-1}.
		C’est considéré comme une erreur d’indiquer un indice
		qui ne correspond pas à une case du tableau (trop petit ou trop grand).
		Par exemple, si on déclare~:
		\lda{tabEntiers : {\Array{100}{entiers}}}	
		il est interdit d’utiliser \lda{tabEntiers[-1]} ou
		\lda{tabEntiers[100]}. 

		\begin{Exercice}{Déclarer et initialiser}
			Écrire un algorithme qui déclare un tableau de 100 chaines
			et met votre nom dans la 3\ieme{} case du tableau.
		\end{Exercice}
	
		\begin{Exercice}{Initialiser un tableau à son indice}
			Écrire un algorithme qui déclare un tableau de 100 entiers
			et initialise chaque élément à la valeur de son indice.
			Ainsi, la case numéro $i$ contiendra la valeur $i$.
		\end{Exercice}

		\begin{Exercice}{Initialiser un tableau aux valeurs de 1 à 100}
			Écrire un algorithme qui déclare un tableau de 100 entiers
			et y met les nombres de 1 à 100.
		\end{Exercice}

		\begin{Exercice}{Compréhension}
			Expliquez la différence entre \lda{tab[i] \Gets tab[i+1]}
			et \lda{tab[i] \Gets tab[i]+1}.
		\end{Exercice}
		
	% ==================
	\section{Initialisation compacte}
	% ==================

		Chaque élément d’un tableau
		doit être manié avec la même précaution 
		qu’une variable simple, 
		c’est-à-dire qu’on ne peut utiliser un élément du tableau 
		qui n’aurait pas été préalablement affecté ou initialisé.
%
		L’initialisation d’un tableau peut se faire 
		via une série d’assignations, case par case
		mais on va aussi se permettre une notation compacte%
		\footnote{%
			Peu de langages de programmation 
			ont une notation compacte de ce genre. 
			Java en possède une version moins puissante
			qui ne permet de coder que le premier des deux exemples.
			C’est pourquoi, en début d’apprentissage, 
			on vous demandera souvent d’initialiser des tableaux
			sans utiliser cette notation compacte.
		}~:
		
		\begin{LDA}
			\Let tab \Gets \{2, 3, 5, 7\} 
			\RComment{équivalent à tab[0] \Gets 2, tab[1] \Gets 3, tab[2] \Gets 5, tab[3] \Gets 7}
			\Let tab \Gets \{0, \dots, 0\} 
			\RComment{tous les éléments du tableau sont initialisés à 0}
		\end{LDA}

	% ==============================
	\section{Tableau et paramètres}
	% ==============================
	
		Le type \emph{tableau} étant un type à part entière,
		il est tout-à-fait éligible comme type
		pour les paramètres et la valeur de retour d’un algorithme.
		Voyons cela en détail.

		\subsection{Passer un tableau en paramètre}
		%------------------------------------------
		
			Les trois sortes de passages de paramètres sont possibles.
			Elles servent à spécifier ce qui sera fait avec
			les cases du tableau.
			Plus précisément~:
			\begin{itemize}
			\item \In : 
				indique que l’algorithme va consulter les valeurs 
				du tableau reçu en paramètre. 
				Les éléments doivent donc avoir été initialisés
				avant d’appeler l’algorithme. Exemple~:
			
				\begin{LDA}
					\LComment{Affiche les éléments d’un tableau}
					\Algo{afficher}{\Par{tab\In}{\Array{10}{entiers}}}{} 
						\For{i}{0}{9}
							\Write tab[i]
						\EndFor
					\EndAlgo 
		
					\Empty
					\LComment{Utilisation possible}
					\Decl{monTab}{\Array{10}{entiers}}
					\Let monTab \Gets \{2,3,5,7,11,13,17,19,23,29\}
					\Stmt afficher(monTab)
				\end{LDA}
				
				Rappelons qu’il s’agit du passage
				par défaut si aucune flèche n’est indiquée.
				
			\item \In\Out :
				indique que l’algorithme va consulter/modifier les valeurs 
				du tableau reçu en paramètre. Exemple~:
			
				\begin{LDA}
					\LComment{Inverse le signe des éléments du tableau}
					\Algo{inverserSigne}{\Par{tab\In\Out}{\Array{10}{entiers}}}{} 
						\For{i}{0}{9}
							\Let tab[i] \Gets -tab[i]
						\EndFor
					\EndAlgo 
		
					\Empty
					\LComment{Utilisation possible}
					\Decl{monTab}{\Array{10}{entiers}}
					\Let monTab \Gets \{2,-3,5,-7,11,13,17,-19,23,29\}
					\Stmt inverserSigne(monTab)
				\end{LDA}
	
			\item \Out :
				indique que l’algorithme va assigner des valeurs 
				au tableau reçu en paramètre.
				Les éléments de ce tableau n’ont donc pas à être initialisés
				avant d’appeler cet algorithme. Exemple~:
			
				\begin{LDA}
					\LComment{Initialise un tableau reçu en paramètre}
					\Algo{initialiser}{\Par{tab\Out}{\Array{10}{entiers}}}{}
						\For{i}{0}{9}
							\Let tab[i] \Gets i
						\EndFor
					\EndAlgo 
					\Empty
					\LComment{Utilisation possible}
					\Decl{monTab}{\Array{10}{entiers}}
					\Stmt initialiser(monTab)
				\end{LDA}

			\end{itemize}

		\subsection{Retourner un tableau}
		%---------------------------------
			
			Comme pour n’importe quel autre type,
			un algorithme peut retourner un tableau.
			Ce sera à lui de le déclarer et de lui donner des valeurs.

			\textbf{Exemple~:}
			
			\begin{LDA}
				\LComment{Retourne un tableau initialisé de 10 entiers}
				\Algo{créer}{}{\Array{10}{entiers}}
					\Decl{tab}{\Array{10}{entiers}}
					\For{i}{0}{9}
						\Let tab[i] \Gets i
					\EndFor
					\Return tab
				\EndAlgo 
				\Empty
				\LComment{Utilisation possible}
				\Decl{monTab}{\Array{10}{entiers}}
				\Let monTab \Gets créer()
			\end{LDA}
		
		\subsection{Paramétrer la taille}
		%---------------------------------
		
			Un tableau peut être passé en paramètre à un algorithme 
			mais qu’en est-il de sa taille~? 
			Plus haut, nous avons écrit un algorithme qui affiche
			un tableau de 10 entiers.
			Il serait utile de pouvoir appeler le même algorithme avec des
			tableaux de tailles différentes. 
			Pour permettre cela, la taille du tableau reçu en paramètre 
			peut être déclarée avec un nom (habituellement \emph{n})
			qui peut être considéré comme un paramètre entrant
			et peut être utilisé dans l’algorithme.
			Idem pour un tableau en retour.
		
			\textbf{Exemple~:}
			
			\begin{LDA}
				\LComment{Affiche un tableau de taille quelconque}
				\Algo{afficher}{\Par{tab}{\Array{n}{entiers}}}{}
					\Write "Tableau de ", n, " éléments".
					\For{i}{0}{n-1} \RComment{Attention ! le dernier élément est à l’indice n-1}
						\Write tab[i]
					\EndFor
				\EndAlgo
				\Empty
				\LComment Crée un tableau d’entiers de taille n, l’initialise à 0 et le retourne.
				\Algo{créer}{n : entier}{\Array{n}{entiers}}
					\Decl{tab}{\Array{n}{entiers}}
					\For{i}{0}{n-1}
						\Let tab[i] \Gets 0
					\EndFor
					\Return tab
				\EndAlgo
				\Empty
				\Algo{test}{}{}
					\Decl{entiers}{\Array{20}{entiers}}
					\Let entiers \Gets créer(20)
					\Stmt afficher(entiers)
				\EndAlgo
			\end{LDA}
			
			La fiche \vref{fiche:tab-passage-param} récapitule tout ça.

			\begin{Exercice}{Trouver les entêtes}
				Écrire les entêtes (et uniquement les entêtes)
				des algorithmes qui résolvent les problèmes suivants~:
				\begin{enumerate}[label=\alph*)]
				\item
					Écrire un algorithme qui 
					inverse le signe de tous les éléments négatifs dans un tableau d’entiers.
				\item
					Écrire un algorithme qui
					donne le nombre d’éléments négatifs dans un tableau d’entiers.
				\item
					Écrire un algorithme qui
					détermine si un tableau d’entiers contient au moins un nombre négatif.
				\item
					Écrire un algorithme qui
					détermine si un tableau de chaines contient
					une chaine donnée en paramètre.
				\item
					Écrire un algorithme qui
					détermine si un tableau de chaines contient
					au moins deux occurrences de la même chaine,
					quelle qu’elle soit.
				\item
					Écrire un algorithme qui 
					retourne un tableau donnant les $n$ premiers nombres premiers,
					où $n$ est un paramètre de l’algorithme.
				\item
					Écrire un algorithme qui 
					reçoit un tableau d’entiers
					et retourne un tableau de booléens de la même taille
					où la case $i$ indique si oui ou non
					le nombre reçu dans la case $i$ est strictement positif.
				\end{enumerate}
			\end{Exercice}
		
	% ==============================================
	\section{Parcours d’un tableau} 
	\label{Les parcours de tableaux}
	% ==============================================

		Dans la plupart des problèmes que vous rencontrerez
		vous serez amené à parcourir un tableau.
		Il est important de maitriser ce parcours.
		Examinons les situations courantes 
		et voyons quelles solutions conviennent.
	
		\subsection{Parcours complet}
		%----------------------------
		
			Vous avez déjà vu dans les exemples
			comment parcourir tous les éléments d’un tableau.
			On s’en sort facilement avec une boucle \K{pour}.
			La fiche \vref{fiche:tab-parcours-complet}
			décrit comment afficher tous les éléments d’un tableau.
			On pourrait faire autre chose avec ces éléments~:
			les sommer, les comparer\dots
				
			\begin{Exercice}{Inverser le signe des éléments}
				Écrire un algorithme qui 
				inverse le signe de tous les éléments négatifs dans un tableau d’entiers.
			\end{Exercice}

			\begin{Exercice}{Somme}
				\marginicon{java}
				Écrire un algorithme qui reçoit en paramètre le tableau
				\lda{tabEnt} de $n$ entiers 
				et qui retourne la somme de ses éléments.
			\end{Exercice}

			\begin{Exercice}{Nombre d’éléments d’un tableau}
				Écrire un algorithme qui reçoit en paramètre le tableau
				\lda{tabRéels} de $n$ réels et qui
				retourne le nombre d’éléments du tableau.
			\end{Exercice}
			
			\begin{Exercice}{Compter les éléments négatifs}
				Écrire un algorithme qui
				donne le nombre d’éléments négatifs dans un tableau d’entiers.
			\end{Exercice}

		\subsection{Parcours partiel}
		%----------------------------
		
			Parfois, on ne doit pas forcément parcourir 
			le tableau jusqu’au bout
			mais on pourra s’arrêter prématurément 
			si une certaine condition est remplie.
			Par exemple~:
			\begin{itemize}
			\item 
				on cherche la présence d’un élément et on vient de le trouver~;
			\item 
				on vérifie qu’il n’y a pas de $0$ et on vient d’en trouver un~;
			\item
				on vérifie que tous les éléments sont positifs
				et on vient d’en trouver un qui est négatif ou nul.
			\end{itemize}
	
			Pour résoudre ce problème,
			on peut partir du parcours complet et~:
			\begin{itemize}
			\item
				Transformer le \K{pour} en \K{tant que}.
			\item
				Introduire un test d’arrêt.
			\end{itemize}
			La fiche \vref{fiche:tab-parcours-partiel}
			détaille un parcours partiel.
				
			\begin{Exercice}{Tester la présence de négatifs}
				Écrire un algorithme qui
				détermine si un tableau d’entiers contient au moins un nombre négatif.
			\end{Exercice}

			\begin{Exercice}{Y a-t-il un pilote dans l’avion~?}
				\marginicon{java}
				Écrire un algorithme qui reçoit en paramètre le tableau
				\lda{avion} de $n$ chaines 
				et qui retourne un booléen 
				indiquant s’il contient au moins un élément de 
				valeur "pilote". 
			\end{Exercice}
		
	% ==============================================
	\section{Taille logique\index{taille logique} et taille physique\index{taille physique}} 
	% ==============================================

		Parfois, on ne sait pas exactement 
		quelle taille doit avoir un tableau.
		Imaginons, par exemple,
		qu’il s’agisse de demander des valeurs à l’utilisateur
		et de les stocker dans un tableau pour un traitement ultérieur.
		Supposons que l’utilisateur va indiquer la fin des données
		par une valeur sentinelle.
		Impossible de savoir, à priori, combien de valeurs il va entrer
		et, par conséquent, la taille à donner au tableau.
		
		Une solution est de créer un tableau suffisamment grand
		pour tous les cas%
		\footnote{%
			En tout cas, 
			suffisamment grand pour tous les cas qu’on accepte
			de prendre en compte; il faudra bien fixer une limite.
		} 
		quitte à n’en n’utiliser qu’une partie.
		Mais comment savoir quelle est la partie du tableau
		qui est effectivement utilisée~?
		Comprenez bien qu’il \textbf{n’y a pas de concept de case vide}.
		Rien ne différencie une case utilisée d’une case non utilisée.
		On s’en sort en stockant les valeurs 
		dans la partie basse du tableau (à gauche)
		et en retenant, dans une variable,
		le nombre de cases effectivement utilisées.
			
		\marginicon{definition}
		La \textbf{taille physique} d’un tableau 
		est le nombre de cases qu’il contient.
		Sa \textbf{taille logique}
		est le nombre de cases actuellement utilisées.
	
		\paragraph{Exemple.}
		Voici un tableau qui ne contient pour l’instant que quatre nombres.
		\begin{center}
			\begin{tabular}{*{10}{|>{\centering\arraybackslash}m{5mm}}|}
				\hline
				10 & 4 & 3 & 7 & ? & ? & ? & ? & ? & ? \\
				\hline
			\end{tabular}
			\\\medskip
			\lda{taillePhysique = 10}
			\qquad
			\lda{tailleLogique = 4}
		\end{center}
		Les cases indiquées par "?" ne sont pas vides~;
		elles peuvent être non initialisées
		(et on ne peut pas tester si une case est non initialisée)
		ou bien contenir une valeur qui n’est pas pertinente.
		
		\paragraph{Exemple.}
		L’algorithme suivant 
		demande des valeurs positives à l’utilisateur 
		et les stocke dans un tableau (maximum 1000).
		Toute valeur négative ou nulle est une valeur sentinelle.
		\begin{LDA}
			\Algo{stockerValeurs}{}{}
				\Decl{tab}{\Array{1000}{entiers}}
				\Decl{nbÉléments}{entier} \RComment{C’est la taille logique}
				\Decl{valeur}{entier}
				\Let nbÉléments \Gets 0
				\Read valeur
				\While{valeur > 0 ET nbÉléments < 1000}
					\Let tab[nbÉléments] \Gets valeur
					\Let nbÉléments \Gets nbÉléments + 1
					\Read valeur
				\EndWhile
				\If{valeur > 0}		\RComment{Pourquoi tester valeur et pas nbÉléments~?}
					\Write "La limite physique a été atteinte"
				\EndIf
			\EndAlgo
		\end{LDA}

		\begin{Exercice}{Modifier l’exemple}
			Tel quel, l’exemple ci-dessus n’est pas très intéressant.
			Ce serait mieux s’il retournait le tableau 
			ce qui permettrait effectivement un traitement ultérieur
			des valeurs.
			Modifiez-le en ce sens.
		\end{Exercice}
		
	% ==============================================
	\section{Des tableaux qui ne commencent pas à 0} 
	% ==============================================

		Les tableaux que nous avons vus jusqu’à présent
		\emph{commencent à 0}
		ce qui signifie que la première case est d’indice 0.
		Il peut être intéressant d’utiliser des tableaux
		qui commencent à un autre indice.
		
		\paragraph{Exemple.}
		Imaginons l’algorithme qui permet de convertir un numéro
		de jour en son intitulé~: 1 donne "lundi", 2 donne "mardi",
		\dots{}
		Une solution possible, sans tableau, serait~:
		
		\begin{LDA}
			\Algo{intituléJour}{\Par{numéroJour}{entier}}{chaine}
				\Switch{numéroJour}
				\Case{1} \algorithmicreturn{} "lundi"
				\Case{2} \algorithmicreturn{} "mardi"
				\Case{3} \algorithmicreturn{} "mercredi"
				\Case{4} \algorithmicreturn{} "jeudi"
				\Case{5} \algorithmicreturn{} "vendredi"
				\Case{6} \algorithmicreturn{} "samedi"
				\Case{7} \algorithmicreturn{} "dimanche"
				\EndSwitch
			\EndAlgo
		\end{LDA}
		
		Mais une solution plus élégante passe par l’utilisation d’un tableau.
		Appelons-le \lda{nomsJours} et stockons-y les noms des jours
		comme illustré ci-dessous.

		\begin{center}
			\begin{tabular}{*{7}{>{\centering\arraybackslash}m{15mm}}}
				1 & 2 & 3 & 4 & 5 & 6 & 7
				\\\hline
				\multicolumn{1}{|>{\centering\arraybackslash}m{15mm}|}{"lundi"} &
				\multicolumn{1}{>{\centering\arraybackslash}m{15mm}|}{"mardi"} &
				\multicolumn{1}{>{\centering\arraybackslash}m{15mm}|}{"mercredi"} &
				\multicolumn{1}{>{\centering\arraybackslash}m{15mm}|}{"jeudi"} &
				\multicolumn{1}{>{\centering\arraybackslash}m{15mm}|}{"vendredi"} &
				\multicolumn{1}{>{\centering\arraybackslash}m{15mm}|}{"samedi"} &
				\multicolumn{1}{>{\centering\arraybackslash}m{15mm}|}{"dimanche"}
				\\\hline
			\end{tabular}
		\end{center}

		Pour obtenir le nom d’un jour, il suffit d’écrire~: 
		\lda{nomsJours[numéroJour]}.
		
		On voit dans cet exemple
		qu’il est plus naturel que le tableau commence à 1.
		Pour déclarer un tel tableau, 
		nous utiliserons la notation suivante~:
		
		\begin{LDA}
			\Decl{nomsJours}{\Array[1]{7}{chaines}}
		\end{LDA}
		
		L’algorithme devient~:
		\begin{LDA}
			\Algo{intituléJour}{\Par{numéroJour}{entier}}{chaine}
				\Decl{nomsJours}{\Array[1]{7}{chaines}}
				\Let nomsJours \Gets \{"lundi", "mardi", "mercredi", "jeudi", "vendredi", "samedi", "dimanche"\}
				\Return nomsJours[numéroJour]
			\EndAlgo
		\end{LDA}
		
		\paragraph{Comment traduire ça dans un langage de programmation~?}
		Certains langages comme Pascal
		permettent de choisir l’indice de début d’un tableau.
		D’autres l’imposent, parfois à 1 comme Cobol 
		mais le plus souvent à 0 comme Java.
		Que faire dans ce cas~? Il existe deux stratégies.
		
		\begin{enumerate}
		\item
			La première passe par une conversion d’indice.
			On stocke les données dans le tableau 
			comme nous l’impose le langage
			
				\begin{tabular}{*{7}{>{\centering\arraybackslash}m{15mm}}}
					0 & 1 & 2 & 3 & 4 & 5 & 6
					\\\hline
					\multicolumn{1}{|>{\centering\arraybackslash}m{15mm}|}{"lundi"} &
					\multicolumn{1}{>{\centering\arraybackslash}m{15mm}|}{"mardi"} &
					\multicolumn{1}{>{\centering\arraybackslash}m{15mm}|}{"mercredi"} &
					\multicolumn{1}{>{\centering\arraybackslash}m{15mm}|}{"jeudi"} &
					\multicolumn{1}{>{\centering\arraybackslash}m{15mm}|}{"vendredi"} &
					\multicolumn{1}{>{\centering\arraybackslash}m{15mm}|}{"samedi"} &
					\multicolumn{1}{>{\centering\arraybackslash}m{15mm}|}{"dimanche"}
					\\\hline
				\end{tabular}
				
			et on accède à l’intitulé du jour $i$ dans la case $i-1$.
			Ce qui donne
	
			\begin{LDA}
				\Algo{intituléJour}{\Par{numéroJour}{entier}}{chaine}
					\Decl{nomsJours}{\Array{7}{chaines}}
					\Let nomsJours \Gets \{"lundi", "mardi", "mercredi", "jeudi", "vendredi", "samedi", "dimanche"\}
					\Return nomsJours[numéroJour-1]
				\EndAlgo
			\end{LDA}
			\bigskip
			
		\item
			La seconde approche crée un tableau avec une case en plus
			et n’utilise pas la première.
				
				\begin{tabular}{>{\centering\arraybackslash}m{2mm}*{7}{>{\centering\arraybackslash}m{15mm}}}
					0 & 1 & 2 & 3 & 4 & 5 & 6 & 7
					\\\hline
					\multicolumn{1}{|>{\centering\arraybackslash}m{2mm}|}{} &
					\multicolumn{1}{>{\centering\arraybackslash}m{15mm}|}{"lundi"} &
					\multicolumn{1}{>{\centering\arraybackslash}m{15mm}|}{"mardi"} &
					\multicolumn{1}{>{\centering\arraybackslash}m{15mm}|}{"mercredi"} &
					\multicolumn{1}{>{\centering\arraybackslash}m{15mm}|}{"jeudi"} &
					\multicolumn{1}{>{\centering\arraybackslash}m{15mm}|}{"vendredi"} &
					\multicolumn{1}{>{\centering\arraybackslash}m{15mm}|}{"samedi"} &
					\multicolumn{1}{>{\centering\arraybackslash}m{15mm}|}{"dimanche"}
					\\\hline
				\end{tabular}
				
			L’algorithme devient
	
			\begin{LDA}
				\Algo{intituléJour}{\Par{numéroJour}{entier}}{chaine}
					\Decl{nomsJours}{\Array{8}{chaines}}
					\Let nomsJours \Gets \{"", "lundi", "mardi", "mercredi", "jeudi", "vendredi", "samedi", "dimanche"\}
					\Return nomsJours[numéroJour]
				\EndAlgo
			\end{LDA}
		
		\end{enumerate}
		
		Cette seconde approche est à préférer 
		car elle modifie le moins l’algorithme de départ 
		et entrainera moins d’erreurs de conversion
		(au prix d’un faible gaspillage de la mémoire).
		Mais elle n’est pas toujours possible.
		Par exemple, imaginons qu’on veuille prendre 
		des relevés de températures (arrondies à l’entier) à Bruxelles 
		et retenir le nombre de fois que chaque température a été relevée.
		Le plus évident serait d’utiliser un tableau allant de -30 à 40
		où la case $i$ donne le nombre de fois 
		que la température $i$ a été relevée.
		Pour cet exemple, impossible d’utiliser la seconde approche
		mais la première est toujours possible en stockant
		le nombre de relevés de la température $i$ dans la case $i+30$.

		\begin{Exercice}{Lancers de dé}
			Écrire un algorithme qui lance $n$ fois un dé
			et compte le nombre de fois que chaque valeur apparait.
			\begin{LDA}
			\Entete{lancersDé}{\Par{n}{entier}}{\Array[1]{6}{entiers}}
			\end{LDA}
		\end{Exercice}
		
		\begin{Exercice}{Nombre de jours dans un mois}
				\marginicon{java}
			Écrire un algorithme qui reçoit un numéro de mois (de 1 à 12)
			ainsi qu’une année et donne le nombre de jours dans ce mois
			(en tenant compte des années bissextiles).
			N’hésitez pas à réutiliser des algorithmes déjà écrits.
		\end{Exercice}
		
		
