%======================
\chapter{Exercices sur les tableaux}
%======================

	Voici quelques exercices
	qui reprennent les différentes notions vues sur les tableaux.
	Pour chacun d’entre-eux~:
	\begin{itemize}
	\item
		demandez-vous quel est le problème le plus proche déjà
		rencontré et voyez comment adapter la solution~;
	\item
		indiquez la complexité de votre solution~;
	\item
		décomposez votre solution en plusieurs algorithmes 
		si ça peut améliorer la lisibilité~;
	\item
		testez votre solution dans le cas général
		et dans les éventuels cas particuliers.
	\end{itemize}
	
	\begin{Exercice}{Renverser un tableau}
		\marginicon{java}
		Écrire un algorithme qui reçoit en paramètre 
		le tableau \lda{tabCar} de $n$ caractères, 
		et qui «~renverse~» ce tableau, 
		c’est-à-dire qui permute le premier élément avec le dernier, 
		le deuxième élément avec l’avant-dernier et ainsi de suite.
	\end{Exercice}
	
	\begin{Exercice}{Tableau symétrique~?}
		Écrire un algorithme qui reçoit en paramètre 
		le tableau \lda{tabChaines} de $n$ chaines 
		et qui vérifie si ce tableau est symétrique, 
		c’est-à-dire si le premier élément est identique au dernier, 
		le deuxième à l’avant-dernier et ainsi de suite.
	\end{Exercice}
		
	\begin{Exercice}{Maximum/minimum}
		\marginicon{java}
		Écrire un algorithme qui reçoit en paramètre le tableau
		\lda{tabEnt} de $n$ entiers et qui
		retourne la plus grande valeur de ce tableau. Idem pour le minimum.
	\end{Exercice}
	
	\begin{Exercice}{Indice du maximum/minimum}
		\label{ex:indiceminmax}
		\marginicon{java}
		Écrire un algorithme qui reçoit en paramètre 
		le tableau \lda{tabEnt} de $n$ entiers 
		et qui retourne l’indice de l’élément contenant 
		la plus grande valeur de ce tableau. 
		En cas d’ex-æquo, c’est l’indice le plus petit qui sera renvoyé.
		
		Que faut-il changer pour renvoyer l’indice le plus grand~?
		Et pour retourner l’indice du minimum~? 
		Réécrire l’algorithme de l’exercice précédent en utilisant celui-ci.
	\end{Exercice}
		
	\begin{Exercice}{Tableau ordonné~?}
		\marginicon{java}
		Écrire un algorithme qui reçoit en paramètre 
		le tableau \lda{valeurs} de $n$ entiers 
		et qui vérifie si ce tableau est ordonné 
		(strictement) croissant sur les valeurs. 
		L’algorithme retournera \lda{vrai} si le tableau est ordonné,
		\lda{faux} sinon.
	\end{Exercice}
	
	\begin{Exercice}{Positions du minimum}
		\marginicon{java}
		Écrire un algorithme qui reçoit en paramètre le tableau
		\lda{cotes} de $n$ entiers et qui
		affiche le ou les indice(s) des éléments 
		contenant la valeur minimale du tableau.
	
		\begin{enumerate}[label=\alph*)]
		\item 
			Écrire une première version «~classique~» avec deux parcours de tableau
		\item
			Écrire une deuxième version qui ne parcourt qu’une seule fois 
			\lda{cotes} en
			stockant dans un deuxième tableau (de quelle taille~?)
			les indices du plus petit élément
			rencontrés (ce tableau étant à chaque fois réinitialisé lorsqu’un
			nouveau minimum est rencontré)
		\item
			Écrire une troisième version qui \textbf{retourne} 
			le tableau contenant les indices.
			Écrire également un algorithme 
			qui appelle cette version puis affiche les indices reçus. 
		\end{enumerate}
	\end{Exercice}

	\begin{Exercice}{Lancers de deux dés}
		Écrire un algorithme qui lance $n$ fois deux dés
		et compte le nombre de fois que chaque somme apparait.
		\begin{LDA}
		\Entete{lancers2Dés}{\Par{n}{entier}}{\Array[2]{12}{entiers}}
		\end{LDA}
	\end{Exercice}
	
	\begin{Exercice}{Occurrence des chiffres}
		\marginicon{java}
		Écrire un algorithme qui reçoit 
		un nombre entier positif ou nul en paramètre
		et qui retourne un tableau de 10 entiers
		indiquant, pour chacun de ses chiffres, 
		le nombre de fois qu’il apparait dans ce nombre. 
		Ainsi, pour le nombre 10502851125, on retournera
		le tableau \{2,3,2,0,0,3,0,0,1,0\}.
	\end{Exercice}
	
	\begin{Exercice}{Les doublons}
		\marginicon{java}
		Écrire un algorithme qui vérifie 
		si un tableau de chaines
		contient au moins 2 éléments égaux.
	\end{Exercice}	
	\bigskip
	
	\begin{Exercice}{Le crible d’Ératosthène}
		\begin{quote}
			\og{} Le crible d’Ératosthène est un procédé 
			qui permet de trouver tous les nombres premiers inférieurs 
			à un certain entier naturel donné N.
			L’algorithme procède par élimination~: 
			il s’agit de supprimer d’une table des entiers de 2 à N 
			tous les multiples d’un entier. 
			En supprimant tous les multiples, 
			à la fin il ne restera que les entiers qui ne sont multiples d’aucun entier, 
			et qui sont donc les nombres premiers.
			On commence par rayer les multiples de 2, 
			puis à chaque fois on raye les multiples du plus petit entier restant.
			On peut s’arrêter lorsque le carré du plus petit entier restant 
			est supérieur au plus grand entier restant, car dans ce cas, 
			tous les non-premiers ont déjà été rayés précédemment.\fg{}
			(source~: Wikipédia)
		\end{quote}
		Le tableau dont il est question peut être un simple tableau
		de booléens; le booléen en position "i" indiquant 
		si le nombre "i" est premier ou pas.

		Écrire un algorithme qui reçoit un entier $n$
		et affiche tous les entiers premiers de $1$ à $n$.
	\end{Exercice}
	
	\begin{Exercice}{Mastermind}
		\marginicon{java}
		Dans le jeu du Mastermind, 
		un joueur A doit trouver une combinaison de k pions de couleur, 
		choisie et tenue secrète par un autre joueur B. 
		Cette combinaison peut contenir éventuellement des pions de même couleur. 
		À chaque proposition du joueur A, 
		le joueur B indique le nombre de pions de la proposition 
		qui sont corrects et bien placés 
		et le nombre de pions corrects mais mal placés. 
		
		Les propositions du joueur A, 
		ainsi que la combinaison secrète du joueur B
		sont contenues dans des tableaux de k composantes de type chaine.
		
		Écrire l’algorithme suivant qui renvoie dans les variables
		\lda{bienPlacés} et \lda{malPlacés}
		respectivement le nombre de pions bien placés et mal placés 
		dans la «~proposition~» du joueur A en la comparant 
		à la «~solution~» cachée du joueur B.
	
		\begin{LDA}
		\Entete{testerProposition}{
			\Par{proposition\In, solution\In}{\Array{n}{chaines}}, 
			\\\hfill 
			\Par{bienPlacés\Out, malPlacés\Out}{entiers}}{}
		\end{LDA}
	\end{Exercice}
	
