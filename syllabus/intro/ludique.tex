% ===============================================
\chapter{Une approche ludique~: Code Studio}
\index{Code Studio}
% ===============================================

	\begin{wrapfigure}{l}{14mm}
	\vskip-4mm
	\includegraphics[scale=0.2]{image/codeorg-studio-logo.png}
	\vskip-2mm
	\end{wrapfigure}
	
	Il existe de nombreux programmes 
	qui permettent de s’initier à la création d’algorithmes.
	Nous voudrions mettre en avant le projet \emph{Code Studio}.
	Soutenu par des grands noms de l’informatique 
	comme \textsf{Google}, \textsf{Microsoft}, 
	\textsf{Facebook} et \textsf{Twitter},
	il permet de s’initier aux concepts de base
	au travers d’exercices ludiques faisant intervenir 
	des personnages issus de jeux que les jeunes connaissent bien 
	comme \textsf{Angry birds} ou \textsf{Plantes et zombies}.
	
	Sur le site \url{http://studio.code.org/} 
	nous avons sélectionné pour vous~:

	\begin{itemize}
	\item
		\textbf{L’heure de code}~: 
		\url{http://studio.code.org/hoc/1}.
		
		Un survol des notions fondamentales en une heure
		au travers de vidéos explicatives et d’exercices interactifs.
	\item
		\textbf{Cours d’introduction}~: 
		\url{http://studio.code.org/s/20-hour}.
		
		Un cours de 20 heures destiné aux adolescents.
		Il reprend et approfondi les éléments effleurés dans
		\og\ L’heure de code\fg	
	\end{itemize}
	
	Nous vous conseillons de créer un compte sur le site
	ainsi vous pourrez retenir votre progression
	et reprendre rapidement votre travail là où vous l’avez
	interrompu.
	
	Votre professeur va vous guider dans votre apprentissage
	pendant le cours et vous pourrez approfondir à la maison.
	
