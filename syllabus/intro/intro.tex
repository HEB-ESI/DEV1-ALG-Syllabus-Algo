% =====================================
\chapter{Résoudre des problèmes}
% =====================================

	\begin{Exergue}
		«~L’algorithmique est le permis de conduire de l’informatique.
		Sans elle, il n’est pas concevable d’exploiter sans risque un ordinateur.~»
		\footnote{[CORMEN e.a., Algorithmique, Paris, Edit. Dunod, 2010, (Cours, 
		exercices et problèmes), p. V] }
	\end{Exergue}

	\marginicon{objectif}
	Ce chapitre a pour but
	de vous faire comprendre ce qu’est une 
	\emph{procédure de résolution de problèmes}.

	%------------------------------
	\section{La notion de problème}
	%------------------------------
	
		\subsection{Préliminaires~:~utilité de l’ordinateur}
		%---------------------------------------------------
		
			L’ordinateur est une machine. 
			Mais une machine intéressante dans la mesure 
			où elle est destinée d’une part, 
			à nous décharger d’une multitude de tâches peu valorisantes, 
			rébarbatives telles que le travail administratif répétitif, 
			mais surtout parce qu’elle est capable de nous aider, 
			voire nous remplacer, dans des tâches plus ardues 
			qu’il nous serait impossible de résoudre sans son existence 
			(conquête spatiale, prévision météorologique, jeux vidéo\dots).
			
			En première approche, 
			nous pourrions dire que l’ordinateur 
			est destiné à nous remplacer, 
			à faire à notre place 
			(plus rapidement et probablement avec moins d’erreurs) 
			un travail nécessaire à la résolution de \textbf{problèmes} 
			auxquels nous devons faire face. 
			Attention~! Il s’agit bien de résoudre des \textit{problèmes} 
			et non des mystères (celui de l’existence, par exemple). 
			Il faut que la question à laquelle
			on souhaite répondre soit \textbf{accessible à la raison}.
	
		\subsection{Poser le problème}
		%---------------------------------------------------
		
			Un préalable à l’activité de résolution d’un problème 
			est de bien \textbf{définir} d’abord 
			quel est le problème posé, 
			en quoi il consiste exactement ; 
			par exemple, faire un baba au rhum, 
			réussir une année d’études, 
			résoudre une équation mathématique\dots
			
			Un problème bien posé doit mentionner 
			l’\textbf{objectif à atteindre},
			c’est-à-dire la situation d’arrivée, 
			le but escompté, le résultat attendu. 
			Généralement, tout problème se définit d’abord explicitement
			par ce que l’on souhaite obtenir.
			
			La formulation d’un problème ne serait pas complète 
			sans la connaissance
			\textbf{du cadre dans lequel se pose le problème~:}
			de quoi dispose-t-on, quelles sont les hypothèses de base, 
			quelle est la situation de départ~? 
			Faire un baba au rhum est un problème tout à fait différent 
			s’il faut le faire en plein désert 
			ou dans une cuisine super équipée~! 
			D’ailleurs, dans certains cas, 
			la première phase de la résolution d’un problème 
			consiste à mettre à sa disposition 
			les éléments nécessaires à sa résolution~:~
			dans notre exemple, 
			ce serait se procurer les ingrédients 
			et les ustensiles de cuisine.
		
			Un problème ne sera véritablement bien spécifié 
			que s’il s’inscrit dans le schéma suivant~:
			
			% Voir si j’en fais un environnement.
			\begin{center}
			\begin{Ovalbox}
				{\textbf{étant donné} [la situation de départ] 
				\textbf{on demande} [l’objectif]}
			\end{Ovalbox}
			\end{center}
		
			Parfois, la première étape dans la résolution d’un problème 
			est de préciser ce problème à partir d’un énoncé flou~:~
			il ne s’agit pas nécessairement d’un travail facile~!
	
			\begin{Emphase}
				\paragraph{Exercice.} Un problème flou.\\
				Soit le problème suivant~:~
				«~Calculer la moyenne de nombres entiers.~».
				\\Qu’est-ce qui vous parait flou dans cet énoncé~?
			\end{Emphase}
			
			Une fois le problème correctement posé, 
			on passe à la recherche et la description 
			d’une \textbf{méthode/procédure de résolution}, 
			afin de savoir comment faire 
			pour atteindre l’objectif demandé 
			à partir de ce qui est donné. 
			Le \textbf{nom} donné à une méthode de résolution 
			varie en fonction du cadre dans lequel se pose le problème~:~
			\textit{façon de procéder, mode d’emploi, marche à suivre, 
			guide, patron, modèle, recette de cuisine, 
			méthode ou plan de travail, algorithme mathématique, 
			programme, directives d’utilisation\dots}
	
	%--------------------------------
	\section{Procédure de résolution}
	%--------------------------------
	
		Une \textbf{procédure de résolution} est une description 
		en termes compréhensibles par l’exécutant 
		de la \textbf{marche à suivre} 
		pour résoudre un problème donné.
		
		On trouve beaucoup d’exemples dans la vie courante~:
		recette de cuisine, mode d’emploi d’un GSM, 
		description d’un itinéraire, 
		plan de montage d’un jeu de construction, etc. 
		Il est clair qu’il y a une infinité de rédactions possibles 
		de ces différentes marches à suivre. 
		Certaines pourraient être plus précises que d’autres,
		d’autres par contre pourraient s’avérer exagérément explicatives.		

		Des différents exemples de procédures de résolution 
		se dégagent les caractéristiques suivantes~:	
		\begin{itemize}
		\item 
			toutes ont un \textbf{nom}
		\item 
			elles s’expriment dans un \textbf{langage}
			(français, anglais, dessins\dots)
		\item 
			l’ensemble de la procédure consiste 
			en une \textbf{série chronologique}
			d’instructions ou de phrases (parfois numérotées)
		\item 
			une instruction se caractérise par un ordre, 
			une action à accomplir,
			une \textbf{opération} à exécuter 
			sur les \textbf{données} du problème
		\item 
			certaines phrases justifient ou expliquent ce qui se passe~:~
			ce sont des \textbf{commentaires}.
		\end{itemize}
	
		On pourra donc définir, en première approche, 
		une procédure de résolution comme un texte, 
		écrit dans un certain langage, 
		qui décrit une suite d’actions à exécuter dans un ordre précis, 
		ces actions opérant sur des objets issus des données du problème.
	
		\subsection{Chronologie des opérations}
		%--------------------------------------
		
			Pour ce qui concerne l’ordinateur, 
			le travail d’exécution d’une marche à suivre 
			est impérativement \textbf{séquentiel}. 
			C’est-à-dire que les instructions d’une procédure de résolution 
			sont exécutées \textbf{une et une seule fois} 
			dans l’ordre où elles apparaissent.
			Cependant certains artifices d’écriture 
			permettent de \textbf{répéter} l’exécution d’opérations 
			ou de la \textbf{conditionner}
			(c’est-à-dire de choisir si l’exécution aura lieu oui ou non 
			en fonction de la réalisation d’une condition).
	
		\subsection{Les opérations élémentaires}
		%---------------------------------------
		
			Dans la description d’une marche à suivre, 
			la plupart des opérations sont introduites par un \textbf{verbe}
			(\textit{remplir}, \textit{verser, prendre, peler}, etc.). 
			L’exécutant ne pourra exécuter une action que s’il la comprend~:
			cette action doit, pour lui, être une action élémentaire, 
			une action qu’il peut réaliser 
			sans qu’on ne doive lui donner des explications complémentaires. 
			Ce genre d’opération élémentaire est appelée \textbf{primitive}.
			
			Ce concept est évidement relatif 
			à ce qu’un exécutant est capable de réaliser. 
			Cette capacité, il la possède d’abord 
			parce qu’il est \textbf{construit} d’une certaine façon 
			(capacité innée). 
			Ensuite parce que, par construction aussi, 
			il est doté d’une faculté d’\textbf{apprentissage} 
			lui permettant d’assimiler, petit à petit, 
			des procédures non élémentaires qu’il exécute souvent. 
			Une opération non élémentaire 
			pourra devenir une primitive un peu plus tard.
			
		\subsection{Les opérations bien définies}
		%----------------------------------------
		
			Il arrive de trouver dans certaines marches à suivre 
			des opérations qui peuvent dépendre d’une certaine manière 
			de l’appréciation de l’exécutant. 
			Par exemple, dans une recette de cuisine on pourrait lire~:~
			\textit{ajouter} \textbf{\textit{un peu}} 
			\textit{de vinaigre, saler et poivrer} 
			\textbf{\textit{à volonté}}, \textit{laisser cuire une} 
			\textbf{\textit{bonne}}
			\textit{ heure dans un four}
			\textbf{\textit{bien}} \textit{chaud, etc.}
			
			Des instructions floues de ce genre 
			sont dangereuses à faire figurer dans une bonne marche à suivre 
			car elles font appel à une appréciation arbitraire de l’exécutant. 
			Le résultat obtenu risque d’être imprévisible 
			d’une exécution à l’autre. 
			De plus, les termes du type \textit{environ, beaucoup, pas trop} 
			et \textit{à peu près} sont intraduisibles 
			et proscrites au niveau d’un langage informatique~!%
			\footnote{%
				Le lecteur intéressé découvrira 
				dans la littérature spécialisée 
				que même les procédures de génération de nombres aléatoires
				sont elles aussi issues d’algorithmes mathématiques 
				tout à fait déterminés.
			}
			
			Une \textbf{opération bien définie} 
			est donc une opération débarrassée
			de tout vocabulaire flou 
			et dont le résultat est \textbf{entièrement prévisible}. 
			Des versions «~bien définies~» des exemples ci-dessus
			pourraient être~:~
			\textit{ajouter 2~cl de vinaigre, ajouter 5~g de sel
			et 1~g de poivre, 
			laisser cuire 65~minutes dans un four chauffé à 220\degre{}C, etc.}
	
			Afin de mettre en évidence la difficulté d’écrire une
			marche à suivre claire et non ambigüe, on vous propose
			l’expérience suivante.
	
			\begin{Emphase}
				\paragraph{Expérience.} Le dessin.
				
				Cette expérience s’effectue en groupe.
				Le but est de faire un dessin 
				et de permettre à une autre personne, 
				qui ne l’a pas vu, 
				de le reproduire fidèlement, 
				au travers d’une «~marche à suivre~».
	
				\begin{enumerate}
				\item
					Chaque personne prend une feuille de papier et 
					y dessine quelque chose en quelques traits précis. 
					Le dessin ne doit pas être trop compliqué ; 
					on ne teste pas ici vos talents de dessinateur~! 
					(ça peut être une maison, une voiture\dots)
				\item
					Sur une \textbf{autre} feuille de papier, 
					chacun rédige des instructions permettant de 
					reproduire fidèlement son propre dessin. 
					Attention~! Il est important de ne
					\textbf{jamais faire référence à la signification du dessin}. 
					Ainsi, on peut écrire~:~«~dessine un rond~» 
					mais certainement pas~:~«~dessine une roue~».
				\item
					Chacun cache à présent son propre dessin et échange 
					sa feuille d’instructions avec celle de quelqu’un d’autre.
				\item
					Chacun s’efforce ensuite 
					de reproduire le dessin d’un autre 
					en suivant \textbf{scrupuleusement} 
					les instructions indiquées sur la feuille reçue en échange, 
					\textbf{sans tenter d’initiative} 
					(par exemple en croyant avoir compris ce
					qu’il faut dessiner).
				\item
					Nous examinerons enfin les différences 
					entre l’original et la reproduction 
					et nous tenterons de comprendre 
					pourquoi elles se sont produites 
					(par imprécision des instructions 
					ou par mauvaise interprétation de celles-ci 
					par le dessinateur\dots)
				\end{enumerate}
			\end{Emphase}
	
			\marginicon{reflexion}
			Quelles réflexions cette expérience vous inspire-t-elle~?
			Quelle analogie voyez-vous 
			avec une marche à suivre donnée à un ordinateur~?
	
			Dans cette expérience, 
			nous imposons que la «~marche à suivre~» 
			ne mentionne aucun mot expliquant le sens du dessin 
			(mettre «~rond~» et pas «~roue~» par exemple). 
			Pourquoi, à votre avis, avons-nous imposé cette contrainte~?
	
		\subsection{Opérations soumises à une condition}
		%-----------------------------------------------
		
			En français, 
			l’utilisation de conjonctions ou locutions conjonctives 
			du type \textit{si}, \textit{selon que}, \textit{au cas où}\dots{}
			présuppose la possibilité 
			de ne pas exécuter certaines opérations 
			en fonction de certains événements. 
			D’une fois à l’autre, 
			certaines de ses parties seront ou non exécutées.
			
			\textbf{Exemple~:}~
			Si la viande est surgelée, 
			la décongeler à l’aide du four à micro-ondes.
	
		\subsection{Opérations à répéter}
		%--------------------------------
		
			De la même manière, 
			il est possible d’exprimer en français une exécution
			répétitive d’opérations en utilisant les mots \textit{tous},
			\textit{chaque}, \textit{tant que}, \textit{jusqu’à ce que},
			\textit{chaque fois que}, \textit{aussi longtemps que}, 
			\textit{faire x fois}\dots 
			
			Dans certains cas, 
			le nombre de répétitions est connu à l’avance
			(\textit{répéter 10 fois}) 
			ou déterminé par une durée 
			(\textit{faire cuire pendant 30 minutes}) 
			et dans d’autres cas il est inconnu.
			Dans ce cas, la fin de la période de répétition 
			d’un bloc d’opérations dépend alors 
			de la réalisation d’une condition 
			(\textit{lancer le dé jusqu’à ce qu’il tombe sur 6}, 
			\textit{faire cuire jusqu’à évaporation complète} \dots). 
			C’est ici que réside le danger de boucle infinie, 
			due à une mauvaise formulation de la condition d’arrêt.
			Par exemple~:~
			\textit{lancer le dé jusqu’à ce que le point obtenu soit 7}\dots{} 
			Bien sûr, 
			un humain doté d’intelligence comprend 
			que la condition est impossible à réaliser, 
			mais un robot appliquant cette directive à la lettre 
			lancera le dé perpétuellement\dots
	
		\subsection{À propos des données}
		%--------------------------------
		
			Les types d’objets figurant 
			dans les diverses procédures de résolution
			sont fonction du cadre 
			dans lequel s’inscrivent ces procédures, 
			du domaine d’application de ces marches à suivre. 
			Par exemple, pour une recette de cuisine, 
			ce sont les ingrédients. 
			Pour un jeu de construction ce sont les briques.
			
			L’ordinateur, quant à lui, manipule principalement
			des données numériques et textuelles. 
			Nous verrons plus tard 
			comment on peut combiner ces données élémentaires 
			pour obtenir des données plus complexes.
	
	%-------------------
	\section{Ressources}
	%-------------------
	
		Pour prolonger votre réflexion sur le concept d’algorithme
		nous vous proposons quelques ressources en ligne~:
		\begin{itemize}
		\item
			Les Sépas 18 - Les algorithmes~:
			\url{https://www.youtube.com/watch?v=hG9Jty7P6Es}
		\item
			Les Sépas 11 - Un bug~:
			\url{https://www.youtube.com/watch?v=deI0GV5sWTY}
		\item
			Le crépier psycho-rigide comme algorithme~:
			\url{https://pixees.fr/?p=446}
		\item
			Le baseball multicouleur comme algorithme~:
			\url{https://pixees.fr/?p=450}
		\item
			Le jeu de Nim comme algorithme~:
			\url{https://pixees.fr/?p=443}
		\end{itemize}
		
