\chapter{Bonnes (et mauvaises) pratiques}

	Au fil des années,
	nous voyons souvent les étudiants écrire
	les mêmes bouts de code qui sont discutables,
	voire à proscrire.
	Discutons ici des plus courants en rappelant
	que le critère qui nous guide est la lisibilité
	et donc la facilité de compréhension de l'algorithme
	ce qui induit évidemment un aspect subjectif, voir culturel%
	\footnote{%
		Certaines communautés de programmeurs habitués à un langage
		adoptent des styles de programmation qui ne sont pas
		partagés par d'autres programmeurs, habitués à un autre langage.
	}.
	
%===================================================
\section{Tester un booléen}
%===================================================

	Lorsqu’il s’agit de comparer deux nombres dans la condition
	d’un \K{si}, vous écrivez tous assez naturellement~:
	\lda{\K{si} (nb1<nb2) \dots}
	\\Il ne vous viendrait pas à l’idée d’écrire~:
	\lda{\K{si} (nb1<nb2 = vrai) \dots}
	
	\bigskip
	\begin{minipage}{9cm}
		Avec un booléen, les choses sont moins simples.
		Par exemple, si on doit faire quelque chose
		si la variable \lda{adulte} est vraie,
		vous êtes nombreux à écrire
		le bout de code ci-contre,
		en tout cas en début d’année.
	\end{minipage}
	\quad
	\begin{minipage}{4cm}
		\begin{LDA}
		\If{adulte = vrai}
			\Stmt \dots
		\EndIf
		\end{LDA}
	\end{minipage}
	\hskip-5mm
	\includegraphics[width=1cm]{icon/dont}
	
	\bigskip
	\begin{minipage}{9cm}
		Cette écriture est correcte mais inutilement lourde.
		La variable étant déjà un booléen qui vaut vrai ou faux,
		il suffit d’écrire~:
	\end{minipage}
	\quad
	\begin{minipage}{4cm}
		\begin{LDA}
		\If{adulte}
			\Stmt \dots
		\EndIf
		\end{LDA}
	\end{minipage}
	\hskip-5mm
	\includegraphics[width=1cm]{icon/do}
	
	\bigskip
	\begin{minipage}{9cm}
		De même, si le test doit être vrai quand la variable booléenne
		est fausse, il suffit d’écrire~: 
	\end{minipage}
	\quad
	\begin{minipage}{4cm}	
		\begin{LDA}
		\If{NON adulte}
			\Stmt \dots
		\EndIf
		\end{LDA}
	\end{minipage}
	\hskip-5mm
	\includegraphics[width=1cm]{icon/do}

	\bigskip
	On comprend que la première écriture vous vienne
	d’abord et on la tolérera les premières séances
	mais vous devrez vite vous en débarrasser~;
	elle ne sera pas tolérée lors des évaluations.
	
%===================================================
\section{Assigner un booléen}\label{B-ass-bool}
%===================================================

	Lorsqu’il s’agit de donner une valeur à un booléen,
	on vous voit souvent écrire un si-sinon.
	Ce n’est pas la bonne idée.

	\bigskip
	\begin{minipage}{9cm}
		Par exemple,
		si la variable booléenne \lda{nul} 
		doit valoir \lda{vrai} si un nombre est égal à 0
		et \lda{faux} sinon,
		vous êtes nombreux à écrire le bout de code ci-contre.		
	\end{minipage}
	\quad
	\begin{minipage}{4cm}
		\begin{LDA}
		\If{nb = 0}
			\Let nul \Gets vrai
		\Else
			\Let nul \Gets faux
		\EndIf
		\end{LDA}
	\end{minipage}
	\hskip-5mm
	\includegraphics[width=1cm]{icon/dont}

	\bigskip
	\begin{minipage}{9cm}
		C’est correct mais inutilement compliqué.
		Puisque on assigne \lda{vrai} si la condition
		est vraie et \lda{faux} si la condition est
		fausse, il suffit d’assigner la condition.
		Ce qui donne~:
	\end{minipage}
	\quad
	\begin{minipage}{4cm}
		\begin{LDA}
		\Let nul \Gets nb = 0
		\end{LDA}
	\end{minipage}
	\hskip-5mm
	\includegraphics[width=1cm]{icon/do}

	\bigskip
	À nouveau,
	on tolérera le si-sinon les premières séances
	mais on attend de vous que vous passiez rapidement
	à une simple assignation.
	Ce sera sanctionné lors des évaluations.

%===================================================
\section{Assigner une valeur en fonction d’une condition}\label{B-ass-val}
%===================================================

	\begin{minipage}{9cm}
		Examinons l’exemple ci-contre 
		qui assigne un \lda{prix}
		qui doit être différent en fonction
		du droit ou non à un tarif réduit.  
	\end{minipage}
	\quad
	\begin{minipage}{4cm}
		\begin{LDA}
		\If{tarifRéduit}
			\Let prix \Gets 8 
		\Else
			\Let prix \Gets 12 
		\EndIf
		\end{LDA}
	\end{minipage}
	\hskip-5mm
	\includegraphics[width=1cm]{icon/do}

	\begin{minipage}{9cm}
		Nous voyons régulièrement
		des étudiants écrire plutôt fièrement
		la version ci-contre
		en arguant qu’elle est plus courte,
		donc mieux.

		Elle comporte effectivement (un peu) moins de lignes
		mais ce critère n’a que très peu d’importance.
		Cette version n’est pas plus rapide
		(elle est même plus lente en cas de tarif réduit)
		et elle est moins lisible car le lecteur
		croit d’abord que le tarif est toujours de 12
		avant de constater qu’il peut être différent.
	\end{minipage}
	\quad
	\begin{minipage}{4cm}
		\begin{LDA}
		\Let prix \Gets 12 
		\If{tarifRéduit}
			\Let prix \Gets 8 
		\EndIf
		\end{LDA}
	\end{minipage}
	\hskip-5mm
	\includegraphics[width=1cm]{icon/dont}

	On vous demande de vous en tenir à la première version.
	La seconde sera sanctionnée.
	
%===================================================
\section{Interrompre une boucle pour}\label{B-for-break}
%===================================================

	On voit trop souvent des étudiants écrire une boucle
	\og{}pour\fg{} et l’interrompre anticipativement en modifiant l’indice; 
	c’est une très mauvaise idée.

	\begin{minipage}{6cm}
		Dans l’exemple ci-contre,
		si une certain condition est vraie,
		on donne une grande valeur à \lda{i}
		pour sortir de la boucle.
	\end{minipage}
	\quad
	\begin{minipage}[t]{7cm}
		\vskip-8mm
		\begin{LDA}
		\For{i}{1}{n}
			\Stmt \dots
			\If{uneCondition}
				\Let i \Gets n+1 \Comment {Pour interrompre la boucle}
			\EndIf
			\Stmt \dots
		\EndFor
		\end{LDA}
	\end{minipage}
	\hskip-5mm
	\includegraphics[width=1cm]{icon/dont}

	Il est \textbf{totalement} interdit de modifier explicitement 
	l’indice d’une boucle \lda{\algorithmicfor}, 
	quelle qu’en soit la raison.
	Ce n’est pas une bonne idée en terme de lisibilité
	et dans certains langages c’est interdit par le compilateur ou
	cela ne fait pas ce qui est attendu.
	
	\begin{minipage}{6cm}
	Si vous devez interrompre un parcours répétitif, 
	lorsqu'une condition est remplie, nous tenons absolument 
	à ce que vous passiez par une boucle tant-que
	et l’utilisation d’un booléen pour contrôler la fin.
	Par exemple~:
	\end{minipage}
	\quad
	\begin{minipage}[t]{7cm}
		\vskip-8mm
		\begin{LDA}
		\Let i \Gets 1
		\Let fini \Gets faux
		\While{i$\leq$n ET NON fini}
			\Stmt \dots
			\If{uneCondition}
				\Let fini \Gets vrai
			\EndIf
			\Stmt \dots
			\Let i \Gets i+1
		\EndWhile
		\end{LDA}
	\end{minipage}
	\hskip-5mm
	\includegraphics[width=1cm]{icon/do}

	\bigskip
	\begin{minipage}{6cm}
	Si le test est la première instruction de la boucle
	et que la condition est courte,
	on peut aussi se passer de la variable booléenne.
	\end{minipage}
	\quad
	\begin{minipage}[t]{7cm}
		\vskip-8mm
		\begin{LDA}
		\Let i \Gets 1
		\While{i$\leq$n ET NON uneCondition}
			\Stmt \dots
			\Let i \Gets i+1
		\EndWhile
		\end{LDA}
	\end{minipage}
	\hskip-5mm
	\includegraphics[width=1cm]{icon/do}

	\medskip
	Si vous tenez absolument à interrompre la boucle \og{}pour\fg{},
	vous pouvez utiliser un \lda{retourner} comme expliqué au point suivant.
	 
%===================================================
\section{Plusieurs retourner}
%===================================================
	
	Dans le cours,
	nous vous avons donné comme règle~:
	\og{}Un seul \lda{retourner} par algorithme\fg{}.

	Cette règle n’est pas partagée par tous les programmeurs.
	Il existe un courant qui tolère plusieurs \lda{retourner}
	dans certains cas précis.

	\begin{minipage}{7cm}
		Un cas concret est l'arrêt anticipé d'une boucle.
		Certains écrivent~:
	\end{minipage}
	\quad
	\begin{minipage}[t]{6cm}
		\vskip-5mm
		\begin{LDA}
			\For{i}{1}{n}
				\If{\dots}
					\Return quelquechose
				\EndIf
			\EndFor
			\Return autrechose
		\end{LDA}
	\end{minipage}
	\hskip-5mm
	\includegraphics[width=1cm]{icon/caution}
	
	\bigskip
	\begin{minipage}{7cm}
		Une autre situation est lorsque la valeur à retourner dépend d’un test.
		On voit alors des bouts de code qui ressemblent à~:
	\end{minipage}
	\quad
	\begin{minipage}[t]{6cm}
		\vskip-5mm
		\begin{LDA}
			\If{\dots}
				\Return quelquechose
			\Else
				\Return autrechose
			\EndIf
		\end{LDA}
	\end{minipage}
	\hskip-5mm
	\includegraphics[width=1cm]{icon/caution}
	
	\medskip
	Ces écritures seront tolérées dans ce cours
	(et donc non sanctionnées)
	mais nous ne les encourageons pas
	par crainte d’abus~: elles ne peuvent se
	justifier que pour des algorithmes très courts.
	
