%=================================
\chapter{Exercices récapitulatifs}
%=================================

	La plupart des exercices qui suivent sont inspirés
	d'anciens examens.
	Ils sont assez conséquents et reprennent
	la plupart des notions vues dans ce cours.

	%=================================	
	\section{AEBBCLRS}
	%=================================

		Le Scrabble est un jeu de lettres, 
		où le but est de former des mots ayant le score le plus élevé possible.
		Pour calculer le score d'un mot, 
		une valeur est associée à chaque lettre de l'alphabet. 
		Par exemple la lettre \texttt{A} vaut 1, \texttt{V} vaut 4, 
		\texttt{J} vaut 8,\dots{}
		Le score d'un mot est la somme de la valeur de ses lettres%
		\footnote{dans le vrai jeu c'est un peu plus compliqué.}.

		Par exemple le score du mot \texttt{JAVA} est $8+1+4+1 = 14$.

		Les joueurs tirent chacun 7 lettres qu'ils placent sur un \emph{chevalet}.
		Le but du jeu est de former des mots à partir des lettres du chevalet et de les placer sur le plateau de jeu.
	
		\subsection*{Valeur des lettres}
	
			\'Ecrire un algorithme, \texttt{valeurLettre}, 
			qui reçoit un caractère et retourne sa valeur. 
			Les valeurs des lettres sont les suivantes: 
			\begin{itemize}
			\item A,E,I,L,N,O,R,S,T,U : 1 point
			\item D,G,M : 2 points
			\item B,C,P : 3 points
			\item F,H,V : 4 points
			\item J,Q : 8 points
			\item K,W,X,Y,Z : 10 points
			\end{itemize}	

			L'algorithme lance une erreur avec un message adéquat 
			si le caractère passé en paramètre n'est pas une lettre majuscule.  
	
		\subsection*{Score d'un mot}
			
			\'Ecrire un algorithme, \texttt{scoreMot}, 
			qui reçoit une chaîne de caractères, 
			\texttt{mot}, et retourne la valeur de ce mot, c'est-à-dire la somme de la valeur de chacune de ses lettres.
	
			\underline{Exemple}: si l'algorithme reçoit le mot \texttt{JAVA}, il retourne 14 (=8+1+4+1).
	
		\subsection*{Mot possible}
	
			\'Ecrire un algorithme, \texttt{motPossible}, qui reçoit un tableau de caractères, 
			\texttt{chevalet}, et une chaîne de caractères, 
			\texttt{mot}, et retourne vrai si les lettres du chevalet permettent d'obtenir le mot donné, et retourne faux dans le cas contraire. 
	
	
			\underline{Exemple}: si l'algorithme reçoit le tableau \texttt{[A, O, V, G, G, J, L]} 
			et le mot \texttt{"ALGO"}, l'algorithme retourne vrai. 
			Par contre si le mot donné est \texttt{"JAVA"}, 
			alors l'algorithme retourne faux car la lettre \texttt{A} n'apparaît pas 2 fois dans le chevalet.
	
	
			\underline{Aide} : Une solution possible est de faire une copie du chevalet 
			et pour chaque lettre du mot de vérifier que la lettre s'y trouve, 
			si elle s'y trouve de la remplacer par un symbole (par exemple un '-').
			Pour cela on définit un algorithme \texttt{copieChevalet} 
			permettant de copier un chevalet et un algorithme \texttt{indiceLettre} 
			permettant de connaître l'indice d'une lettre dans un chevalet 
			(cet algorithme retourne -1 si la lettre ne s'y trouve pas). 
	
		\subsection*{Meilleur mot}
	
			\'Ecrire un algorithme, \texttt{meilleurMot}, qui reçoit :
			\begin{itemize}
			\item un tableau de caractères, \texttt{chevalet}, qui contient les lettres disponibles pour former un mot ;
			\item un tableau de chaînes de caractères, \texttt{dico}, qui contient par exemple tous les mots du dictionnaire.
			\end{itemize}
			L'algorithme retourne le mot du dictionnaire que l'on peut obtenir avec les lettres du chevalet 
			et qui a le score le plus élevé. 
			Si aucun mot n'est possible avec le chevalet, l'algorithme retourne un mot vide \texttt{""}.
	
	
